
This section seeks to offer preliminary insights into how greenhouse gas (GHG) emissions respond to the implementation of two agricultural public policies: the introduction of tariffs and the provision of subsidies.

To do so, we consider a two-countries market, with a importing country $H$, and exporting country $F$.

We denotes the supply and the demand functions for both country $i \in \{H, F\}$: 
\begin{align*}
S_i &= S_i^0 \left( 1+\eta_i \left( P_i-P^0_i \right)/P_i^0 \right) \\ 
D_i &= D_i^0 \left( 1+\epsilon_i \left( P_i-P_i^0 \right)/P_i^0 \right),
\end{align*}
with $S_i$ and $D_i$, the quantities produced and demanded by country $i$, $P_i$ the price, in country $i$, $\eta_i$ and $\epsilon_i$ are the supply and demand elasticity in country $i$. $X^0$ denotes the initial value of $X$.

Those two countries form the entirety of the economy, hence, the sum of their productions is equal to the sum of their consumptions: $D_H-S_H = S_F-D_F$.

For simplification, we introduce the following aggregate elasticities:
\begin{itemize}
    \item[-] total demand elasticity $\epsilon=\frac{\partial D}{\partial P_F} \frac{P_F^0}{D^0}= \left( \epsilon_H \frac{D_H^0}{P_H^0}+\epsilon_F \frac{D_F^0}{P_F^0} \right)\frac{P_F^0}{D^0} < 0$,
    \item[-] total supply elasticity $\eta=\frac{\partial S}{\partial P_F}\frac{P_F^0}{S^0}=\left( \eta_H \frac{S_H^0}{P_H^0}+\eta_F \frac{S_F^0}{P_F^0} \right)\frac{P_F^0}{S^0} > 0$,
    \item[-] home import demand elasticity $\mu_H= \frac{\partial (D_H-S_H)}{\partial P_H}\frac{P_H^0}{M_H^0}=\frac{\epsilon_H D_H^0-\eta_H S_H^0}{M_H^0}<0$,
    \item[-] foreign export supply elasticity $\chi_F=\frac{\partial (S_F-D_F)}{\partial P_F}\frac{P_F^0}{X_F^0}=\frac{\eta_F S_F^0-\epsilon_F D_F^0}{X_F^0}>0$.
\end{itemize}

For each policies, we will We will examine, for each policies, their effects on total emissions, throughout their effects on international prices (foreign prices), and total production.


\section{Introduction of a tariff in home country}\label{Sec_tariff}

We consider the first policy, where country $H$ introduce a tariff $t$, it implies the following relations between prices: $P_H = P_F + t$.

With, the tariff, the price in the exporting country becomes
$$
\frac{P_F}{P_F^0}= -\frac{\mu_H (1 - t/P_H^0) - \chi_F}{\eta-\epsilon}\frac{X_F^0}{D^0},
$$
and varies negatively with $t$: 
$$
\frac{\partial P_F}{\partial t} = \frac{\mu_H}{\eta - \epsilon}  \frac{X_F^0}{D^0} \frac{P_F^0}{P_H^0}<0.
$$

Total production from both countries is governed by 
$$
Q = S_H^0 + S_F^0 + \frac{(P_H^0 - P_F^0 - t)(S_H^0 \eta_H \chi_F + S_F^0 \eta_F \mu_H)}{P_F^0 \mu_H - P_H^0 \chi_F},
$$
and thus varies according to 
$$
\frac{\partial Q}{\partial t} = \frac{S_H^0 \eta_H \chi_F + S_F^0 \eta_F \mu_H}{P_F^0 \mu_H - P_H^0 \chi_F}.
$$
The sign of the change is equal to the sign of $S_H^0 \eta_H \chi_F + S_F^0 \eta_F \mu_H$. There is no clear effect of a tariff increase on total production: a first (direct) effect increase home production due to tariff increase, while a second (indirect) effect decreases global production because of lower foreign prices. 

Concerning the global emissions $E$, if we consider emissions as the product of quantity product with a factor of emission, we find:
$$
E = E^0 + \frac{(P_H^0 - P_F^0 - t)(E_H^0 \eta_H \chi_F + E_F^0\eta_F\mu_H)}{P_F^0\mu_H - P_H^0\chi_F},
$$ and $$
\frac{\partial E}{\partial t} = \frac{E_H^0 \eta_H \chi_F + E_F^0 \eta_F \mu_H}{P_F^0 \mu_H - P_H^0 \chi_F}.
$$
The sign is the same as $E_H^0 \eta_H \chi_F + E_F^0 \eta_F \mu_H$. Here, again, the effect of increasing tariff on global emissions is ambiguous. The formula is the same as for production except supplied quantities are replaced by emissions. The higher are domestic emissions $E_H^0$, the more likely is a tariff increase to increase emissions.

See Appendix \ref{appendix:intuitions_tariff} for proofs, and particular cases.


\subsection{Provision of a subventions to production in home country}\label{Sec_subvention}

In this subsection, we consider the provision of a subventions to production in the home country, this implies a change in our supply functions, we now have $S_F=S_F^0 \left( 1+\eta_F \left( P_F-P^0_F \right)/P_F^0 \right)$, and with the subvention $s$ $S_H=S_H^0 \left( 1+\eta_H \left( P_H + s - P^0_H \right)/P_H^0 \right)$. For simplification, we consider $P_H = P_F = P$.

Providing a subvention leads to the following price expression and derivate:
\begin{align*}
\frac{P}{P^0} = 1 + \frac{\eta_H}{\mu_H - \chi_F} \frac{s S_H^0}{P^0 X_F^0}, \\
\frac{\partial P}{\partial s} = \frac{\eta_H}{\mu_H - \chi_F} \frac{S_H^0}{X_F^0} < 0.
\end{align*}
This means that introducing a subvention to production in the home country will lower prices, in the home and foreign countries.

Total supply becomes:
\begin{align*}
S &= S^0 + \eta_H S_H^0 \frac{s}{P^0} \left[ 1 - \frac{\eta}{\chi_F - \mu_H} \frac{S^0}{X_F^0} \right], \\
\frac{\partial S}{\partial s} &= \frac{\eta_H S_H}{P^0}\left[ 1 - \frac{\eta}{\chi_F - \mu_H} \frac{S^0}{X_F^0} \right].
\end{align*}
Since $X_F^0(\chi_F - \mu_H) = \eta S^0 - \epsilon D^0$, and $\epsilon < 0$, we have $1 > \frac{\eta}{\chi_F - \mu_H} \frac{S^0}{X_F^0}$, which means that a subvention to the production will increase total production.

With linear emission to supply, we have:
$$ 
E = E^0 + \eta_H \frac{s}{P^0} \left[ E_H^0 - \frac{\eta_H E_H^0 + \eta_F E_F^0}{\chi_F - \mu_H} \frac{S_H^0}{X_F^0} \right],
$$ 
hence 
$$
\frac{\partial E}{\partial s} = \frac{\eta_H}{P^0} \left[ E_H^0 - \frac{\eta_H E_H^0 + \eta_F E_F^0}{\chi_F - \mu_H} \frac{S_H^0}{X_F^0} \right].
$$
This time, the sign of the derivative is more ambiguous: it depends on the relation between $(\eta - \epsilon)E_H^0 S^0$ and $(\eta_H E_H^0 + \eta_F E_F^0) S_H^0$, if the former is higher than the later, then the subvention will lead to more emissions.

See Appendix \ref{appendix:intuitions_subvention} for details.

