
This section offers preliminary insights into how greenhouse gas (GHG) emissions respond to the implementation of two agricultural public policies: the introduction of tariffs and the provision of subsidies. 

To do so we consider two countries market, with a home importing country $H$, and a foreign exporting  country $F$.

We denote the supply and demand functions for both countries, with country $i \in \{H, F\}$, as follows:
$$
S_i = S_i^0\left(1 + \eta_i\frac{P_i - P_i^0}{P_i^0}\right), \qquad
D_i = D_i^0\left(1 + \epsilon_i\frac{P_i - P_i^0}{P_i^0}\right),
$$
where $S_i$ and $D_i$ are the quantities produced and demanded by country $i$, $P_i$ is the price in country $i$, and $\eta_i$ and $\epsilon_i$ are the supply and demand elasticities in country $i$, respectively. Here, $X^0$ denotes the initial value of $X$.

Since these two countries comprise the entire economy, the difference between domestic demand and production in one country equals the difference between production and demand in the other: 
$$
D_H - S_H = S_F - D_F.
$$

For simplicity, we introduce the following aggregate elasticities:
\begin{itemize}
    \item[-] Total demand elasticity
    $$
    \epsilon = \frac{\partial D}{\partial P_F} \frac{P_F^0}{D^0} = \left( \epsilon_H \frac{D_H^0}{P_H^0} + \epsilon_F \frac{D_F^0}{P_F^0} \right)\frac{P_F^0}{D^0} < 0,
    $$
    \item[-] Total supply elasticity
    $$
    \eta = \frac{\partial S}{\partial P_F} \frac{P_F^0}{S^0} = \left( \eta_H \frac{S_H^0}{P_H^0} + \eta_F \frac{S_F^0}{P_F^0} \right)\frac{P_F^0}{S^0} > 0,
    $$
    \item[-] Home import demand elasticity
    $$
    \mu_H = \frac{\partial (D_H - S_H)}{\partial P_H} \frac{P_H^0}{M_H^0} = \frac{\epsilon_H D_H^0 - \eta_H S_H^0}{M_H^0} < 0,
    $$
    \item[-] Foreign export supply elasticity
    $$
    \chi_F = \frac{\partial (S_F - D_F)}{\partial P_F} \frac{P_F^0}{X_F^0} = \frac{\eta_F S_F^0 - \epsilon_F D_F^0}{X_F^0} > 0.
    $$
\end{itemize}

For each policy, we examine its effects on total emissions through its impact on international prices (foreign prices) and total production.

\section{Introduction of a tariff in the home country}\label{Sec_tariff}

Consider the first policy, where country $H$ introduces a tariff $t$. This implies the following relation between prices: 
$$
P_H = P_F + t.
$$

Under the tariff, the price in the exporting country becomes
$$
\frac{P_F}{P_F^0} = -\frac{\mu_H (1 - t/P_H^0) - \chi_F}{\eta - \epsilon}\frac{X_F^0}{D^0},
$$
and it varies negatively with $t$: 
$$
\frac{\partial P_F}{\partial t} = \frac{\mu_H}{\eta - \epsilon} \frac{X_F^0}{D^0} \frac{P_F^0}{P_H^0} < 0.
$$

The total production from both countries is given by 
$$
Q = S_H^0 + S_F^0 + \frac{(P_H^0 - P_F^0 - t)(S_H^0 \eta_H \chi_F + S_F^0 \eta_F \mu_H)}{P_F^0 \mu_H - P_H^0 \chi_F},
$$
and thus varies according to 
$$
\frac{\partial Q}{\partial t} = \frac{S_H^0 \eta_H \chi_F + S_F^0 \eta_F \mu_H}{P_F^0 \mu_H - P_H^0 \chi_F}.
$$
The sign of this change is determined by that of $S_H^0 \eta_H \chi_F + S_F^0 \eta_F \mu_H$. There is no clear effect of a tariff increase on total production: a first (direct) effect increases home production, while a second (indirect) effect decreases global production because of lower foreign prices.

Concerning global emissions $E$, if we consider emissions as the product of the quantity produced and an emission factor, we obtain:
$$
E = E^0 + \frac{(P_H^0 - P_F^0 - t)(E_H^0 \eta_H \chi_F + E_F^0 \eta_F \mu_H)}{P_F^0 \mu_H - P_H^0 \chi_F},
$$
and thus
$$
\frac{\partial E}{\partial t} = \frac{E_H^0 \eta_H \chi_F + E_F^0 \eta_F \mu_H}{P_F^0 \mu_H - P_H^0 \chi_F}.
$$
Here, the sign is the same as that of $E_H^0 \eta_H \chi_F + E_F^0 \eta_F \mu_H$. In other words, the effect of increasing the tariff on global emissions is ambiguous; higher domestic emissions $E_H^0$ increase the likelihood that a tariff hike will raise emissions.

See Appendix \ref{appendix:intuitions_tariff} for proofs and special cases.

\section{Provision of a subsidy to production in the home country}\label{Sec_subvention}

In this subsection, we consider the provision of a subsidy to production in the home country. This changes our supply functions. We now have:
$$
S_F = S_F^0\left(1 + \eta_F\frac{P_F - P_F^0}{P_F^0}\right),
$$
and, with the subsidy $s$,
$$
S_H = S_H^0\left(1 + \eta_H\frac{P_H + s - P_H^0}{P_H^0}\right).
$$
For simplicity, we assume $P_H = P_F = P$.

Providing a subsidy leads to the following price expression and derivative:
\begin{align*}
\frac{P}{P^0} &= 1 + \frac{\eta_H}{\mu_H - \chi_F} \frac{s \, S_H^0}{P^0 X_F^0}, \\
\frac{\partial P}{\partial s} &= \frac{\eta_H}{\mu_H - \chi_F} \frac{S_H^0}{X_F^0} < 0.
\end{align*}
This result indicates that introducing a subsidy for production in the home country will lower prices in both the home and foreign countries.

The total supply becomes:
$$
S = S^0 + \eta_H \frac{s \, S_H^0}{P^0}\left[1 - \frac{\eta}{\chi_F - \mu_H}\frac{S^0}{X_F^0}\right],
$$
and its derivative is
$$
\frac{\partial S}{\partial s} = \frac{\eta_H S_H^0}{P^0}\left[1 - \frac{\eta}{\chi_F - \mu_H}\frac{S^0}{X_F^0}\right].
$$
Given that $X_F^0(\chi_F - \mu_H) = \eta S^0 - \epsilon D^0$ and $\epsilon < 0$, we have
$$
1 > \frac{\eta}{\chi_F - \mu_H}\frac{S^0}{X_F^0},
$$
which implies that a subsidy on production will increase total production.

Assuming emissions are linearly related to supply, we obtain:
$$ 
E = E^0 + \eta_H \frac{s}{P^0}\left[E_H^0 - \frac{\eta_H E_H^0 + \eta_F E_F^0}{\chi_F - \mu_H}\frac{S_H^0}{X_F^0}\right],
$$ 
and therefore
$$
\frac{\partial E}{\partial s} = \frac{\eta_H}{P^0}\left[E_H^0 - \frac{\eta_H E_H^0 + \eta_F E_F^0}{\chi_F - \mu_H}\frac{S_H^0}{X_F^0}\right].
$$
In this case, the sign of the derivative is ambiguous; it depends on the relationship between $(\eta - \epsilon)E_H^0 S^0$ and $(\eta_H E_H^0 + \eta_F E_F^0) S_H^0$. If the former exceeds the latter, the subsidy will lead to increased emissions.

See Appendix \ref{appendix:intuitions_subvention} for further details.

