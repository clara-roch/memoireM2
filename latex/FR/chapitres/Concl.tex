Ce mémoire estime que l'impact de la suppression des droits de douane dans le secteur agricole pourrait réduire les émissions de gaz à effet de serre du secteur de plus de 6~\%. Pour ce faire, nous avons utilisé un modèle d'équilibre partiel de commerce agricole à quatorze régions  —recouvrant l'ensemble des pays — trente biens agricoles — n'excluant que les alcools et les stimulants — et un bien extérieur. Il prend en compte les usages intermédiaires des différents biens, en utilisant des tables d'usage et de production des différents biens à travers des processus agro-industriels et métaboliques. Il utilise également une représentation de l'usage de la terre avec une fonction de gestion multilogit et l'usage d'intrants intensifs. Nous avons également évalué que le retrait des droits de douane sur les seuls produits issus de l'élevage conduisait à une augmentation totale des émissions du secteur de plus de 3~\%. Les différentes simulations nous permettent de dire que les droits de douane affectent les émissions de GES du secteur par plusieurs canaux, d'abord la mise en place de droit de douane peut conduire à une augmentation des coûts de l'alimentation animale, ce qui résulte directement en une augmentation des prix de production, puis à la consommation, conduisant ensuite à une réduction de leur consommation donc de leur production, qui a pour effet de réduire les émissions de GES, ces biens étant les plus émetteurs du secteur. Ensuite, les droits de douane peuvent conduire à directement augmenter les prix à la consommation de ces mêmes biens, ce qui a le même effet. À l'inverse les droits de douane peuvent relativement augmenter le prix d'autres aliments sans significativement réduire le coût de l'alimentation animale et/ou celui des biens issus de l'élevage, cette baisse relative du prix des biens issus de l'élevage, conduira alors à une augmentation des émissions du secteur. Parallèlement, nous avons vu que toutes les régions ne réagissent pas de la même manière à une suppression des droits de douane, les régions importatrices de viande voient leurs émissions locales diminuer, en échange d'une augmentation dans les régions exportatrices. Les régions importatrices voient également leur prix à la consommation diminuer, tandis que les régions exportatrices voient leur prix augmenter. Finalement, les émissions de la consommation des régions importatrices ne diminuent que si les régions vers lesquelles elles exportent la production des biens qu'elles consomment les produisent avec une intensité carbone suffisamment faible pour compenser l'augmentation des consommations.

Ce travail en appelle d'autres, d'abord, nous pourrions constater si la magnitude des effets observés serait similaire en désagrégeant plus encore les régions, ainsi que les processus de production de bien issus de l'élevage, dans notre modèle. Nous pouvons également penser à ajouter les émissions des transports internationaux. Nous pouvons également nous pencher sur l'effet de la prise en compte de la variation des surfaces allouées à l'agriculture dans le modèle. Parallèlement, nous pourrions chercher à trouver quel niveau de droits de douane serait optimal pour réduire les émissions du secteur, sans conduire à une augmentation trop drastique des prix de l'alimentation dans une région. Nous pouvons également chercher à évaluer l'impact des politiques actuelles sur les droits de douane.

Le cœur de notre modèle, nous permet également de regarder l'effet d'autres politiques commerciales, un prochain axe d'étude serait donc de voir l'impact de subventions à la production\footnote{L'annexe \ref{appendix:subventions} expose à la manière du chapitre \ref{intuition} l'effet d'une politique de subvention à la production.} et de la mise en place d'une taxe carbone aux frontières — qui n'est en réalité qu'une sorte de droit de douane amélioré. Dans notre modèle, parmi les émissions liées aux cultures environ un tiers est dû à l'usage de fertilisants, l'autre à l'usage d'énergie et le reste est moins contrôlable. En l'état, notre modèle pourrait faire jouer sur la quantité de fertilisants pour réduire l'intensité carbone des cultures. Les émissions du cheptel étant majoritairement dû aux fonctions métaboliques des animaux, il n'est pas possible de jouer sur son intensité carbone. La mise en place d'une taxe carbone aux frontières devrait donc dans notre modèle en l'état conduire à une diminution de l'usage des entrants, et de l'export de biens issus du secteur de l'élevage, en faveur d'une augmentation des échanges de biens servant à leur alimentation. L'article de \cite{Laborde2020} évalue l'impact de ces deux politiques à l'aide du modèle d'équilibre général MIRAGRODEO de l'IFPRI.
