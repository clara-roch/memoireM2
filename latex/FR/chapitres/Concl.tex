\textbf{blablabla}

+6-7~\% d'émissions en supprimant les droits de douane. Sans les boissons qui sont pas mauvaises non plus en terme d'émissions, mais bon elles représentaient 3~\% des données de bases donc peut-être qu'on aurait vite fait 5~\% de 6 pourcent d'émissions en plus, soi pas grand-chose.


\paragraph{Autres politiques}\label{Sec_subvention}
Nous pouvons aussi nous intéresser aux effets d'autres politiques sur les émissions de GES du secteur. Par exemple, regardons l'effet des subventions à la production sur les émissions de GES. L'annexe \ref{appendix:subventions} expose à la manière du chapitre \ref{intuition} l'effet d'une politique de subvention à la production.

Possibilité de mettre en place une taxe carbone aux frontières, qui est une sorte de droit de douane spécial carbone. Dans notre modèle, parmi les émissions liées aux cultures environ un tiers est dû à l'usage de fertilisants, l'autre à l'usage d'énergie et le reste est moins contrôlable. En l'état, notre modèle pourrait faire jouer sur la quantité de fertilisants pour réduire l'intensité carbone des cultures. Les émissions du cheptel étant majoritairement dû aux fonctions métaboliques des animaux, il n'est pas possible de jouer sur son intensité carbone. La mise en place d'une taxe carbone aux frontières devrait donc avec dans notre modèle en l'état conduire à une diminution de l'usage des entrants. \textbf{ Article de Laborde et al 2020 sur les subventions à la prod et leurs impacts}

\textbf{parler de l'article lu dans le cours de deCara sur les méthodes de prod et em Frank et al, je crois. + trouver article sur taxe carbone aux frontières (+ faire un mini modèle qui donne une intuition là-dessus)}


\textbf{écrire remarque sur les gens qui disent que le commerce c'est cool parce que ça permet d'optimiser les prod, de telle sorte que chaque pays produit ce qu'il fait de mieux, et avec un peu de chance, on vérifie bien que coûte moins cher = utilise moins d'entrant, et les entrants sont émissifs (energie, fertilisants, qtt d'inputs), et donc là avec les droits de douane, on va plus produire chez soit donc on n'optimise pas sur ce point \#rip}