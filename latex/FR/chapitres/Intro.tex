% [Amorce]

L'objectif de ce mémoire, et d'étudier les effets des droits de douane existants sur les émissions de gaz à effet de serre (GES) dans le secteur agricole. Pour se faire nous nous intéresserons à un modèle de commerce de denrées agricoles. Ce modèle est un modèle d'équilibre partiel, basé sur le modèle présenté dans \cite{Gouel2021}, et utilisant les traitements de données permis par la base de données FABIO de \cite{Bruckner2019}, ainsi que des données FAOSTAT et GTAP. Le modèle représente les biens agricoles issus des cultures, de l'élevage et des processus de transformation agro-industrielle, ainsi qu'un bien non-agricole. Ces trois types de biens agricoles, sont régits par des équations de productions différentes~: les cultures suivant une fonction de rendement isoélastique et tandis que leur répartition sur l'espace agricole disponible est régi par une fonction d'entropie multilogit, les deux autres secteurs sont produits à partir des cultures, selon des fonctions de production Léontief. La demande dans ce modèle suit une fonction à élasticité de substitution constante (CES), dépendant entre autre du pays d'origine. Le commerce n'est soumis à aucun pris iceberg, mais à des droits de douane. Nous ne prenons pas en compte les effets de changement d'usage des sols, malgré son impact important sur les émissions de GES, la quantité de terres cultivées est donc absorbée par les prairies qui servent de pâture aux animaux.

Ce modèle permet donc de voir les effets des droits de douanes sur les émissions de GES. D'autres papiers comme \textbf{balabla lit sur les autres papiers qui parlent de ges et droits de douanes}

La littérature existante témoigne aussi de modèle d'équilibre liant agricultures et commerce.




% [politique taxe carbone aux frontières, et généralement pourquoi regarder droit de douanes et émissions de GES]
% [Biblio]
% [Biblio sur les effets des droits de douanes sur les émissions]
% [Biblio sur les modèles climats, agro, commerce]


Pour réaliser cette étude, nous considérons les surfaces agricoles constantes, i.e. nous ne considérons pas les forêts, et la possibilité qu’elles changent de taille ici, c’est-à-dire que nous n’évaluons pas l’impact que l’agriculture peut avoir sur les couverts forestiers et donc sur les émissions de GES liées à leur évolution.

Le reste du mémoire est organisé comme suit. Le chapitre \ref{intuition} propose un modèle simple à deux pays et un produit, afin de comprendre comment l’implémentation de politiques agricoles affecte, par le biais des équilibres de marché, les émissions de GES. Ensuite le chapitre suivant \ref{model} présente quant à lui le modèle d’équilibre partiel utilisé pour mener notre étude, ce modèle inclus de nombreux pays et secteurs. Il se base sur celui présenté dans les papiers de \cite{Gouel2021} et \cite{Gouel2025}, en utilisant, à la place de fonctions de distribution de Fréchet pour capturer l’effet de l’hétérogénéité des cultures sur les rendements, et une fonction de gestion multilogit qui témoigne de l’augmentation des coûts associée à une trop forte ou trop faible spécialisation des cultures, et une fonction isoélastique pour les rendements suivant \cite{Carpentier2013}. Le chapitre \ref{data} décrit les données utilisées ainsi que leurs traitements pour intégration au modèle. Enfin, le chapitre \ref{results} présente les résultats et donc les conclusions sur l’impact des politiques agricoles sur les émissions de GES au travers du commerce en agriculture.
