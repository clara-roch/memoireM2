\textbf{Ce mémoire n'est pas fini, il y a donc encore tout plein des fautes de Français} :-)

Les politiques tarifaires appliquées aux produits agricoles modifient les flux commerciaux, la spécialisation productive, la localisation et l'intensité des émissions de gaz à effet de serre liées à l'agriculture. En protégeant certaines filières ou en restreignant les importations, les droits de douane peuvent avoir des effets indirects sur les émissions qui ne sont pas toujours intuitifs.

L'objectif de ce mémoire est d'étudier les effets des droits de douane existants sur les émissions de gaz à effet de serre (GES) dans le secteur agricole. Pour ce faire, nous nous intéresserons à un modèle de commerce de denrées agricoles. Ce modèle est un modèle d'équilibre partiel, inspiré de celui présenté dans \cite{Gouel2021}, et tirant parti des données issues de la base FABIO de \cite{Bruckner2019}, ainsi que des données FAOSTAT et GTAP. Le modèle représente les biens agricoles issus des cultures, de l'élevage et des processus de transformation agro-industrielle, ainsi qu'un bien non-agricole. Ces trois types de biens agricoles sont régis par des équations de production différentes : les cultures suivent une fonction de rendement isoélastique et leur répartition sur l'espace agricole disponible est déterminée par une fonction d'entropie multilogit, tandis que les deux autres secteurs sont produits à partir des cultures via des fonctions de production de type Léontief. La demande dans ce modèle suit une fonction à élasticité de substitution constante (CES) qui dépend notamment du pays d'origine. Le commerce n'est soumis à aucun coût de transport supplémentaire ("iceberg"), mais à des droits de douane. Nous ne tenons pas compte des effets liés au changement d'usage des sols, malgré leur impact important sur les émissions de GES ; ainsi, la quantité de terres cultivées est absorbée par les prairies servant au pâturage des animaux.

Ce modèle permet d'évaluer les effets des droits de douane sur les émissions de GES dans le secteur agricole. La littérature existante propose diverses approches pour analyser ces impacts, que ce soit via des politiques tarifaires ou des mesures de soutien publiques. Par exemple, \cite{Laborde2020} étudie l’impact des programmes de soutien, en se concentrant sur des dispositifs tels que les subventions à la production, les distorsions de prix aux frontières et les investissements technologiques visant à réduire les émissions de GES dans le secteur agricole. Dans une approche d'équilibre général, l'article de \cite{Shin2025} examine l'influence des régimes tarifaires en s'appuyant sur des équations de gravité régies par des fonctions CES d’Armington, similaires à celles utilisées dans notre modèle, mais sans intégrer les usages de la terre ni les processus de transformation agricole. D'autres études se penchent sur le lien entre politiques tarifaires et émissions de GES dans des contextes différents. Ainsi, \cite{Cary2020} évalue l’impact des politiques tarifaires — et pas uniquement dans le secteur agricole — sur les émissions aux États-Unis, concluant que les droits de douane tendent à réduire ces émissions. Enfin, \cite{Elobeid2021} s'intéresse spécifiquement aux droits de douane imposés par la Chine sur le porc, le soja, le maïs et le blé américains. En couplant le modèle d'équilibre partiel de production agricole CARD avec le modèle entrées-sorties IMPLAN et en intégrant les effets des changements d'affectation des terres à l'échelle mondiale, cette étude conclut à une réduction potentielle des émissions de GES pouvant atteindre jusqu'à 83,7 Mt de CO\textsubscript{2} eq. Par ailleurs, \cite{Himics2018} montre que la libéralisation des échanges agricoles dans l'UE entraîne des effets négatifs sur les émissions en raison de fuites de carbone.

La littérature existante comprend également plusieurs modèles d'équilibre liant agriculture et commerce. Notre modèle s'inspire principalement de \cite{Gouel2021}, qui quantifie le rôle du commerce dans l'adaptation aux changements d'avantages comparatifs induits par le changement climatique, à l'aide d'un modèle commercial d'équilibre général quantitatif. \cite{Gouel202x} présente trois méthodes équivalentes pour modéliser l'allocation des surfaces agricoles : les CET, les distributions de Fréchet des rendements, et les fonctions logit multinomial permettant de régulariser l'entropie de la gestion des terres. D'autres études proposent des approches complémentaires~: \cite{Farrokhi2023} examine les effets de l'adoption de technologies modernes dans l'agriculture et leur interaction avec le commerce, en analysant comment la réduction des barrières commerciales influence leur adoption et impacte la productivité agricole. Cette étude, basée sur un modèle d'équilibre général inspiré de \cite{Costinot2016} et \cite{Sotelo2020}, détermine l'allocation des cultures en fonction des prix de marché. Enfin, le modèle d'équilibre général quantitatif de \cite{CorreaDias2025} évalue les implications environnementales des changements de comportement des consommateurs vers des produits plus intensifs en émissions de GES, ainsi que l'adoption de technologies de production plus émettrices.

Pour réaliser cette étude, nous considérons les surfaces agricoles comme constantes. Autrement dit, nous ne prenons pas en compte les forêts ni la possibilité qu'elles changent de taille, et nous n'évaluons donc pas l’impact que l’agriculture peut avoir sur les couverts forestiers, ainsi que sur les émissions de GES liées à leur évolution.

Le reste du mémoire est organisé comme suit. Le chapitre \ref{intuition} propose un modèle simple à deux pays et un produit, afin de comprendre comment les droits de douane affectent, par le biais des équilibres de marché, les émissions de GES. Ensuite, le chapitre \ref{model} présente, quant à lui, le modèle d’équilibre partiel utilisé pour mener notre étude. Ce modèle s’appuie sur celui présenté dans \cite{Gouel2021} et \cite{Gouel2025}, en substituant les fonctions de distribution de Fréchet par une fonction de gestion multilogit qui traduit l’augmentation des coûts associés à une spécialisation des cultures excessive ou insuffisante, et en recourant à une fonction isoélastique pour les rendements, conformément à \cite{Carpentier2013} et à l’équivalence avec une fonction de Fréchet introduite dans \cite{Gouel202x}. Le chapitre \ref{data} décrit les données utilisées ainsi que leur traitement pour les intégrer au modèle. Enfin, le chapitre \ref{results} présente les résultats, accompagnés d’une analyse rapide de la sensibilité du modèle aux choix des paramètres, et en déduit les conclusions sur l’impact des politiques agricoles sur les émissions de GES au travers du commerce en agriculture.

% [peut-être citer l'article de \cite{Zaho2020}, qui dit qu'un tier des émissions globales de l'argriculture passe par les échanges internationaux.]