\textbf{Ce mémoire n'étant pas fini il y a encore des fautes de Français}

Les politiques tarifaires appliquées aux produits agricoles modifient les flux commerciaux, la spécialisation productive, la localisation et l'intensité des émissions de gaz à effet de serre liées à l'agriculture. En protégeant certaines filières ou en restreignant les importations, les droits de douane peuvent avoir des effets indirects sur les émissions qui ne sont pas toujours intuitifs. \textbf{Plus de blabla.}

L'objectif de ce mémoire est d'étudier les effets des droits de douane existants sur les émissions de gaz à effet de serre (GES) dans le secteur agricole. Pour ce faire, nous nous intéresserons à un modèle de commerce de denrées agricoles. Ce modèle est un modèle d'équilibre partiel, basé sur celui présenté dans \cite{Gouel2021}, et tirant parti des données issues de la base FABIO de \cite{Bruckner2019}, ainsi que des données FAOSTAT et GTAP. Le modèle représente les biens agricoles issus des cultures, de l'élevage et des processus de transformation agro-industrielle, ainsi qu'un bien non-agricole. Ces trois types de biens agricoles sont régis par des équations de production différentes : les cultures suivent une fonction de rendement isoélastique et leur répartition sur l'espace agricole disponible est déterminée par une fonction d'entropie multilogit, tandis que les deux autres secteurs sont produits à partir des cultures via des fonctions de production de type Léontief. La demande dans ce modèle suit une fonction à élasticité de substitution constante (CES) qui dépend notamment du pays d'origine. Le commerce n'est soumis à aucun coût iceberg, mais à des droits de douane. Nous ne tenons pas compte des effets liés au changement d'usage des sols, malgré leur impact important sur les émissions de GES ; ainsi, la quantité de terres cultivées est absorbée par les prairies servant de pâture aux animaux.

Ce modèle permet donc de voir les effets des droits de douane sur les émissions de GES. D'autres papiers comme \textbf{balabla biblio papiers qui parlent de ges et droits de douane}

La littérature existante témoigne aussi de modèle d'équilibre liant agriculture et commerce. \textbf{biblio modèle commerce et ag}


% [politique taxe carbone aux frontières, et généralement pourquoi regarder droit de douane et émissions de GES]
% [Biblio]
% [Biblio sur les effets des droits de douane sur les émissions]
% [Biblio sur les modèles climats, agro, commerce]


Pour réaliser cette étude, nous considérons les surfaces agricoles comme constantes. Autrement dit, nous ne prenons pas en compte les forêts ni la possibilité qu'elles changent de taille, et nous n'évaluons donc pas l’impact que l’agriculture peut avoir sur les couverts forestiers, ainsi que sur les émissions de GES liées à leur évolution.

Le reste du mémoire est organisé comme suit. Le chapitre \ref{intuition} propose un modèle simple à deux pays et un produit, afin de comprendre comment les droits de douane affectent, par le biais des équilibres de marché, les émissions de GES. Ensuite, le chapitre \ref{model} présente, quant à lui, le modèle d’équilibre partiel utilisé pour mener notre étude. Ce modèle s’appuie sur celui présenté dans \cite{Gouel2021} et \cite{Gouel2025}, en substituant les fonctions de distribution de Fréchet par une fonction de gestion multilogit qui traduit l’augmentation des coûts associés à une spécialisation des cultures excessive ou insuffisante, et en recourant à une fonction isoélastique pour les rendements, conformément à \cite{Carpentier2013} et à l’équivalence avec une fonction de Fréchet introduite dans \cite{Gouel202x}. Le chapitre \ref{data} décrit les données utilisées ainsi que leur traitement pour les intégrer au modèle. Enfin, le chapitre \ref{results} présente les résultats, accompagnés d’une analyse rapide de la sensibilité du modèle aux choix des paramètres, et en déduit les conclusions sur l’impact des politiques agricoles sur les émissions de GES au travers du commerce en agriculture.
