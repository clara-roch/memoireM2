[Amorce]

Dans ce mémoire, nous allons étudier les effets des droits de douane sur les émissions de gaz à effet de serres du secteur. Pour se faire nous nous intéresserons à un modèle de commerce de denrées agricoles. Les politiques que nous étudions sont celles des droits de douane.

    [politique taxe carbone aux frontières, et généralement pourquoi regarder droit de douanes et émissions de GES]

    [Biblio]
    [Biblio sur les effets des droits de douanes sur les émissions]
    [Biblio sur les modèles climats, agro, commerce]


Pour réaliser cette étude, nous considérons les surfaces agricoles constantes, i.e. nous ne considérons pas les forêts, et la possibilité qu’elles changent de taille ici, c’est-à-dire que nous n’évaluons pas l’impact que l’agriculture peut avoir sur les couverts forestiers et donc sur les émissions de gaz à effet de serre liées à leur évolution.

Le reste du mémoire est organisé comme suit. Le chapitre \ref{intuition} propose un modèle simple à deux pays et un produit, afin de comprendre comment l’implémentation de politiques agricoles affecte, par le biais des équilibres de marché, les émissions de gaz à effet de serre. Ensuite le chapitre suivant \ref{model} présente quant à lui le modèle d’équilibre partiel utilisé pour mener notre étude, ce modèle inclus de nombreux pays et secteurs. Il se base sur celui présenté dans les papiers de ~\cite{Gouel2021} et ~\cite{Gouel2025}, en utilisant, à la place de fonctions de distribution de Fréchet pour capturer l’effet de l’hétérogénéité des cultures sur les rendements, et une fonction de gestion multilogit qui témoigne de l’augmentation des coûts associée à une trop forte ou trop faible spécialisation des cultures, et une fonction isoélastique pour les rendements suivant ~\cite{Carpentier2013}. Le chapitre \ref{data} décrit les données utilisées ainsi que leurs traitements pour intégration au modèle. Enfin, le chapitre \ref{results} présente les résultats et donc les conclusions sur l’impact des politiques agricoles sur les émissions de gaz à effet de serre au travers du commerce en agriculture.
