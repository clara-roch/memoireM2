\section{Conséquence de la suppression des droits de douane}

Pour comprendre l'effet des droits de douane sur les émissions de GES du secteur, nous partons de la situation actuelle. Nous changeons ensuite de cet équilibre initial les valeurs des droits de douane pour les annuler, le second équilibre atteint, nous pouvons constater les émissions files dans cette situation contrefactuelle et déduire de l'impacte des droits de douane.

La suppression des droits de douane dans notre modèle, avec le paramétrage présenté, augmente les droits de douane de l'ensemble du secteur d'environ 6,5~\%\footnote{Cf. \ref{subsec:vit_cvg}}.

Cherchons à décomposer les différents effets en jeu ici. Dans notre modèle réduit du chapitre \ref{intuition} nous concluions que les émissions varient avec les droits de douane selon \ref{eq:eq_intuition_delta_em}\footnote{$\left({\partial E}\right)/\left( {\partial t} \right) = \left( {E_H^0 \eta_H \chi_F + E_F^0 \eta_F \mu_H} \right)/\left( {P_F^0 \mu_H - P_H^0 \chi_F} \right).$}, c'est-à-dire qu'elles augmentent avec l'introduction d'un droit de douane si les émissions du pays producteurs sont suffisamment basses par rapport à celles du pays mettant le droit de douane en place. Cherchons à montrer ce phénomène, en décomposant notre suppression de droits de douane.

\subsection{Suppression des droits de douane d'un pays pour un bien}\label{subsec:ae_anp}

Choisissons de retirer les droits de douane, d'un pays et d'un bien vers un autre pays moins émetteur sur ce bien (quantité $\times$ potentiel), idéalement ces droits de douane sont élevés. Nous choisissons l'Asie de l'Est comme région importatrice et les deux Amériques et l'Union européenne comme régions exportatrices, car toutes quatre sont respectivement importatrice et exportatrices nettes (cf. figure \ref{fig:sankey2}). On constate également que l'Asie de l'Est importe en valeur beaucoup de produits animaliers. Nous choisissons donc de faire une première décomposition, en retirant les droits de douane à l'import de produits issus de l'élevage en Asie de l'Est.

L'équilibre atteint après avoir changé les droits de douane conduit à une faible réduction de 0,32~\% des émissions à l'échelle mondiale. En regardant le détail par régions, on constate en effet que les émissions de notre région importatrice ont été réduites de 5,2~\%, celles de nos trois régions exportatrices ont elles similairement augmenté de 0,5-0,6~\%.

Concernant les productions, on constate\footnote{Cf. annexe \ref{annexe:ae_anp} pour tableau de résultats} une diminution en Asie de l'Est, des produits animaliers (tous de -5,3~\%\footnote{Les biens sont, dans notre modèle, tous issus du même processus, ils évoluent donc conjointement.}), ainsi que les biens utilisés pour l'alimentation animale (tourteaux -6,8~\%, fourrage -1,6~\%). En parallèle, on observe une réduction de l'import de biens servant (directement ou non) à l'alimentation animale, de 9~\% pour le soja, de 8~\% pour les tourteaux et de 5~\% pour le maïs. Cette réduction est de plus, plus importante pour les aliments venant des trois régions concernées par la suppression des droits de douane. On observe également une très forte réduction de l'import de produits animaliers venant des autres régions, allant pour les produits animaliers de 20~\% (pour la viande rouge) à 62-66~\% (pour le porc) (on observe même une réduction de près de moitié pour le sud-américain), et allant de 10 à 4~\% pour leur alimentation.

Parallèlement, on observe une diminution des prix à la consommation allant de 18~\% à 7~\% pour ces biens, et de moins d'un pourcent pour tous les autres biens. Mais étant donné que la consommation alimentaire totale ne varie que faiblement (+0,5~\%), on observe une légère diminution de la consommation des autres biens. À l'inverse, la consommation totale de produits agricoles diminue dans nos trois autres régions, et ce très légèrement, en parallèle d'une augmentation de moins d'un pourcent des prix à la consommation, des biens de la chaîne de valeur animale. Ces variations sont similaires pour les prix producteurs.

Finalement, on observe un comportement similaire à celui de notre modèle réduit, malgré l'influence du reste du monde qui n'est pas considéré dans notre mini-modèle.


\subsection{Suppression de droits de douane dans tous les pays sur un seul bien}

\subsubsection{Bien individuel non animal}

Ici nous supprimons alternativement les droits de douane pour un bien à la fois. Intuitivement deux choses peuvent se passer~: ou bien le bien est utilisé dans l'alimentation animale ou bien il ne l'est pas. S'il ne l'est pas, ce bien substituera d'autres aliments, dont les animaliers dans l'alimentation humaine, si au contraire il l'est suffisamment, la suppression des droits de douane conduira à une diminution du prix de l'alimentation animale et donc en bout de chaîne des produits animaliers, qui conduira à une augmentation de leur consommation. Si la consommation de produit animalier augmente, on s'attend à ce que les émissions totales augmentent. Parallèlement, pour un bien qui est utilisé majoritairement dans l'alimentation animale, uniquement dans une partie des pays, la suppression des droits de douane, risque d'augmenter le prix de l'alimentation animale uniquement dans ces pays, car les prix nationaux pour ce bien les pays l'exportant vont augmenter. Cet effet pourrait même aller jusqu'à réduire les émissions de GES dans certains pays. Le tableau \ref{tab:res_item} montre l'impact sur les émissions globales de la suppression, des droits de douane lui correspondant. Intuitivement, une autre chose peu se passer, si le bien n'est pas significativement utilisé dans la consommation animale, mais qu'il a un potentiel émissif supérieur à d'autres biens, l'augmentation des émissions due à celle de sa consommation risque de ne pas recouvrir la diminution des émissions du secteur animal suite à la baisse de sa consommation.

\begin{table}[hbt!]
    \centering
    \caption{Conséquence de la suppression des droits de douane liés à chaque bien.}
    \label{tab:res_item}
    \resizebox{\textwidth}{!}{%
        \begin{tabular}{l|ccccccccc}
            \hline
                                                                         & MAI            & RIC\tnote{a}  & WHE           & CAK             & CER             & SOY             & V\&F             & R\&T            & POT              \\ \hline
            $\Delta$ des émissions totales                               & +1,4           & +0,41         & +0,19         & +0,127          & +0,047          & +0,045          & +0,036           & +0,025          & +0,024           \\
            $\Delta$ du prix de l'alimentation animale                   & {[}-21;+3,3{]} & {[}-2,3;+4{]} & {[}-3,8;+5{]} & {[}-1,7;+0,3{]} & {[}-1,2;+1,3{]} & {[}-4,9;+4,1{]} & {[}-0,7;+0,47{]} & {[}-1,1;+0,6{]} & {[}-0,5;+0,04{]} \\
            Nbr de régions avec $\partial E/\partial t >0$               & 5              & 11            & 8             & 7               & 9               & 4               & 8                & 12              & 6                \\
            Nbr de régions avec $\partial P_\text{feed}/ \partial t > 0$ & 11             & 12            & 10            & 8               & 9               & 12              & 6                & 9               & 3                \\ \hline
        \end{tabular}%
    }
    \vspace{0.5em}
    \resizebox{\textwidth}{!}{%
        \begin{tabular}{l|cccc|ccccccc|ccc}
                                                                         & OSD             & SUG             & OCR     & OIL     & BAN     & CIT     & CTS          & PAK     & COT   & OPF   & CTL   & TOM               & SGC     & OILP    \\ \hline
            $\Delta$ des émissions totales                               & -0,111          & -0,09           & -0,017  & -0,008  & $0^-$   & $0^-$   & $0^-$        & $0^-$   & $0^-$ & $0^-$ & $0^-$ & $0^+$             & $0^+$   & $0^+$   \\
            $\Delta$ du prix de l'alimentation animale                   & {[}-1,5;+0,2{]} & {[}-0,3;+1,7{]} & $\sim$0 & $\sim$0 & $\sim$0 & $\sim$0 & {[}0;+0,3{]} & $\sim$0 & 0     & 0     & 0     & {[}-0,18;$0^+${]} & $\sim$0 & $\sim$0 \\
            Nbr de régions avec $\partial E/\partial t >0$               & 10              & 3               & 0       & 9       & 3       & 3       & 0            & 1       & 0     & 0     & 0     & 3                 & 4       & 6       \\
            Nbr de régions avec $\partial P_\text{feed}/ \partial t > 0$ & 11              & 11              & 9       & 4       & 8       & 4       & 13           & 1       & 0     & 0     & 0     & 1                 & 4       & 13
        \end{tabular}%
    }
    \begin{tablenotes}
        \item [a] Pour assurer la convergence de cette simulation, il a fallu d'abord libéraliser les importations n'allant pas en Asie de l'Est, puis réduire progressivement les droits de douane appliqués par la région.
        \item Les colonnes sont classées selon leur impact sur les émissions totales.
        \item Les valeurs telles quelles ne sont pas significatives, nous nous intéressons plutôt au signe.
    \end{tablenotes}
\end{table}

On observe plusieurs choses~:
\begin{itemize}
    \item globalement la suppression des droits de douane conduit à l'augmentation des émissions totales~;
    \item les biens qui sont utilisés presque exclusivement pour la consommation humaine (tomates, bananes, agrumes, huiles, sucres) ou dans une chaîne de valeur ne permettant pas l'alimentation animale (cultures sucrières, les différents états du coton, cœur de palmier), conduisent à une faible réduction totale des émissions~;
    \item la diminution des prix de l'alimentation animale dans des pays concorde avec l'augmentation dans d'autres, conduisant à une augmentation des émissions dans une partie uniquement des pays~;
    \item pour les biens majoritaires dans tous les pays pour l'alimentation animale (maïs, tourteaux et soja), on observe une certaine complémentarité entre les régions, ou le prix de l'alimentation animale est réduite et les émissions du pays augmente ou bien c'est l'inverse~;
    \item les biens servant aussi bien à l'alimentation animale qu'humaine, conduisent à une augmentation totale des émissions, tout en augmentant dans de nombreux pays le prix de l'alimentation animale.
\end{itemize}

Regardons plus en détail ce qui se passe lorsque l'on supprime les droits de douane sur le maïs, le riz. Parallèlement, regardons le cas des oléagineux, qui semblent aller à l'envers des intuitions. L'annexe \ref{annexe:sankey} représente des graphes de Sankey des commerces extérieurs des différents biens que nous analysons ci-dessous.

La libération des échanges de maïs entraîne une diminution prix à la consommation du maïs, en Asie de l'Est, Moyen-Orient, Asie du SE, Chine et Afrique Sud Saharienne, cependant cette diminution implique une baisse du coût moyen de la nourriture animale uniquement en Asie de l'Est, Moyen-Orient et Chine~; on remarque que les deux régions pour lesquelles la baisse du prix du maïs n'entraîne pas une baisse du prix de l'alimentation animale sont des régions pour lesquelles l'augmentation du reste des prix de l'alimentation animale ne permet pas de baisser le panier moyen, en effet ces deux régions consomment désormais plus de céréales, de riz et des tubercules (parmi d'autres), hors les prix de ces denrées ont également augmenté. Concernant les autres régions, tous les prix constituant le panier d'aliments pour l'élevage, il est tout naturel que le prix de l'alimentation augmente. Cela conduit à une diminution du prix moyen des bien agricole uniquement pour l'Asie de l'Est (-4\%), le Moyen-Orient et la Chine ($~0^-$), la plus grande augmentation est ressentie en Amérique du Sud (+0,8~\%). Cependant, seule les cinq régions qui ont ressentie une baisse des prix à la consommation des prix du maïs observent une augmentation totale des émissions~; les variations d'émissions s’étalent de -1,6~\% pour l'Amérique du Sud, jusqu'à +2,1~\% pour l'Asie de l'Est. A l'échelle monde, l'augmentation des émissions est surtout allouée aux maïs (+3,6~\%) dont la demande en fertilisant augmente dans toutes les régions ayant vu le prix du maïs augmenter (jusqu'à +50~\% dans en Russie et Communautés des États indépendants, ou encore +22~\% en Amérique du Sud). Les émissions allouées aux cheptels augmentent également, mais ne sont pas dues à une augmentation de l'usage d'intrants sur les pâtures (qui augmentent au maximum de 1~\% en AmS), mais bien à l'augmentation de la production qui augmentent dans nos cinq régions (plus également légèrement l'Océanie), jusqu'à +22~\% en Asie de l'Est. À l'opposé la réduction maximale de production de viande n'est que de 2~\% en Amérique du Sud.

Pour le riz, qui a comparativement un potentiel émissif plutôt élevé, on observe une diminution des prix à la consommation très forte en Asie de l'Est (-63~\%), puis plus modérée de 8 à 2~\% en Amérique centrale, dans toute l'Europe et la Chine, les prix augmentent de plus de 10~\% en Océanie, Asie du Sud et du SE et Amérique du Nord. L'Asie du Sud et du SE sont les deux plus grandes régions exportatrices de riz. L'Océanie et l'Amérique du Nord, sont quant à elles deux régions ne consommant pas beaucoup de riz, cette augmentation n'est donc pas significative. Pour contrecarrer, ces augmentations de prix à la consommation, toutes les régions à l'exception de l'Asie de l'Est (-22~\%) et de l'Amérique centrale (-4~\%) augmentent la surface allouée à la culture du riz, de +5~\% en Asie du SE, et +15~\% en Asie du Sud, à +70~\% et +207~\% en Amérique du Nord et Océanie. Parallèlement on observe une augmentation allant dans le même sens, mais un peu plus forte de l'utilisation des intrants (i.e. augmentation de l'usage de fertilisants pour une même surface cultivée). Étant donné que les émissions du riz sont avant tout lié à l'usage du sol et de fertilisants (cf. \ref{data}), on explique facilement l'augmentation de +20~\% des émissions de GES lié au riz dans le monde. Parallèlement, on observe une légère augmentation des émissions de GES (de moins de 1~\%) des substituts au riz dans l'alimentation humaine (blé, pommes de terre, céréales, etc.), notamment due à l'augmentation de l'usage d'intrants, permettant de palier à la perte de terre en faveur du riz, mais aussi une diminution des émissions de l'élevage. En effet, on observe une légère augmentation de moins d'un pourcent, des prix de l'alimentation animale dans toutes les régions, sauf en Asie de l'Est (-3~\%), en Amérique centrale ($0^-$), en Asie du Sud et SE (+4~\%), qui entraîne une diminution de la production animale, allant jusqu'à 4~\% en Asie du SE, et donc des émissions qui leur sont allouées.

En retirant droits les droits de douane de tous les pays sur les oléagineux et étant donné qu'ils permettent la production de tourteaux, on s'attendrait à observer une substitution dans les échanges entre les deux, en faveur des oléagineux. \textbf{blabla}


\subsubsection{Produits transformés d'origine animale}

Ici, nous opérons une libération totale des droits de douane sur les produits animaliers, ce qui correspond à une extension du cas vu dans la sous-section \ref{subsec:ae_anp}. Contrairement au cas Asie de l'Est, on observe une augmentation émissions totales de 3,4~\%, on observe une diminution des émissions dans les régions Europe hors-UE, Asie de l'Est, Amérique centrale, toute l'Afrique et la Chine, ces régions sont toutes six des régions importatrices de produits finis d'origine animale, et diminuent de ce fait leur production nationale, leur demande de travail et d'intrants pour l'élevage'. Le prix de l'alimentation animale y varie de -4 à +1,5~\%, mais la production y diminue bien (de -27 à -3~\%). Parallèlement, étant donné que ces régions consomment plus de viande, lait et œufs (jusqu'à +14~\% pour la viande rouge en Europe hors-UE), elle en importe plus (+34~\% pour Europe hors-UE), et importent moins de biens à destination de l'alimentation animale (-30~\% toujours en nUE). Ces réductions sont accompagnées d'une augmentation de l'élevage dans toutes les autres régions (jusqu'à +67~\% en Océanie, ou +9 et +7~\% dans l'UE et en Amérique du Sud, deux grandes régions exportatrices de produits d'origine animale). Leurs imports de tourteaux augmentent, et ce, principalement depuis les six régions qui bénéficient de cette libéralisation. On observe également, que les régions exportatrices de viande et autres, n'augmentent pas la production pour les plantes fourragères, ou bien les intrants. On observe donc une réelle re-répartition des spécialisations, les régions exportatrices de viande se focalisent sur leur production, en important des régions importatrice de viande les biens servant à l'alimentation animale. Finalement, on observe également de faibles diminutions des émissions liées à l'ensemble des biens majoritairement utilisées pour l'alimentation humaine, comme les huiles, les fruits et légumes, ces diminutions témoignent de leur remplacement dans l'alimentation humaine en défaveur des produits d'origine animale.

\subsubsection{D'un produit issu de l'élevage relativement peu émetteur}

Nous supprimons d'abord l'ensemble les droits de douane des produits laitiers, ils sont consommés en quantités importantes dans toutes les régions. L'augmentation résultante est moins importante qu'après la suppression des droits de douane de tout le secteur, mais reste significativement importante (+2,2~\%). Pour ne pas tout re-détailler, nous observons dans les grosses lignes les mêmes effets que lors de la suppression sur tous les produits animaliers. Mais surtout, on observe du fait de notre choix de modélisation des produits issus de l'élevage, que la production augmente de la même manière pour l'ensemble des produits du secteur, chacun étant produit conjointement par une fonction de production Léontief. Ainsi, pour compenser l'augmentation forcée des autres prix animaliers, on observe une diminution des prix à la consommation des autres denrées issus de l'élevage.

À l'inverse lorsque nous retirons les droits de douane sur les œufs, nous observons une réduction de 0,2~\% des émissions de GES. Les réductions, des émissions sont dues cette fois-ci à une réduction plus importante de la production dans certaines régions d'une augmentation d'en d'autres. Les prix des œufs ayant diminué, la consommation est transférée d'autre produits animaliers, vers ces premiers et demande donc une plus faible production pour les autres produits animaliers. Ainsi, malgré la fonction de production conjointe entre les différents produits issus de l'élevage, il est quand même possible d'observer une diminution, quoique faible.


\subsection{Suppression de tous les droits de douane}

Comme on a pu le voir précédemment la suppression des droits de douane, quelqu'ils soient, semble conduire à une diminution du prix de l'alimentation animale ou bien directement des produits animaliers, ce qui conduit à une augmentation de leur consommation, donc à une augmentation des émissions totales.


\section{Sensibilité du modèle}

\subsection{Sensibilité aux paramètres}

\textbf{dis moi si tu pense que ça n'a aucun sens comme changement, plz}
\begin{table}[hbt!]
    \centering
    \caption{Influence de la modification des paramètre sur les résultats finaux.}
    \label{tab:sensibilite_param}
    \begin{tabular}{c|cccc}
        \hline
        Paramètre changé           & Asie de l'Est et produit animalier?? & Maïs & ?? & Libéralisation totale \\ \hline
        $\emptyset$                &                                      &      &    &                       \\
        $\kappa = 0.1$             &                                      &      &    &                       \\
        $\kappa = 2$               &                                      &      &    &                       \\
        $\kappa_\text{feed} = 0.1$ &                                      &      &    &                       \\
        $\kappa_\text{feed} = 2$   &                                      &      &    &                       \\
        $\sigma = 10$              &                                      &      &    &                       \\
        $\varsigma = 1$            & (je sens y va rien se passer)                                            \\
    \end{tabular}%
\end{table}

\subsection{Sensibilité à la vitesse de convergence}\label{subsec:vit_cvg}

Pour faire converger le modèle vers une solution, nous opérons une suite d'équilibres intermédiaires représentant chacun un niveau de douane réduit. Nous ne pouvons, en effet actuellement pas atteindre l'équilibre fil correspondant à une absence de droits de douane directement à partir de l'équilibre initial, le solveur (\textit{mcp} ou \textit{cns} de GAMS\footnote{\textbf{Allez voir si y a pas des trucs écrits dans le cours de CG. Ou dans doc GAMS\dots}}) ne trouve pas de solution. Une première solution consiste à succéder une suite de chocs de -1~\% (ou de -0,5~\%) de droits de douane. Ce résultat nous conduit à une augmentation totale des émissions de 6,7~\%. À l'inverse, une évolution plus brusque de sept chocs de 10~\% jusqu'à -70~\%, puis de chocs de 4~\% jusqu'à la suppression totale des droits de douane conduit à une augmentation de 6,27~\%.

Cette observation témoigne en partie de l'importance du rythme d'application de politique dans leur impact. Ici, un changement plus brutal conduit à une moindre évolution de notre système. \textbf{Trouver des papiers qui parlent du rythme d'implantation, dans son impact.}

\textbf{Faire un graphe montrant en abscisse \% de droits de douane en -, et en ordonnée \% de GES en +, sauf que gams, et je sais pas comment stocker les valeurs intermédiaires d'une boucle. Demander à CG}
