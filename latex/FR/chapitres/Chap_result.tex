\section{Conséquence de la suppression des droits de douane}

Pour comprendre l'effet des droits de douane sur les émissions de GES du secteur, nous partons de la situation actuelle. Nous changeons ensuite de cet équilibre initial les valeurs des droits de douane pour les annuler, le second équilibre atteint, nous pouvons constater les émissions files dans cette situation contrefactuelle et déduire de l'impacte des droits de douane.

La suppression des droits de douane dans notre modèle, avec le paramétrage présenté, augmente les droits de douane de l'ensemble du secteur d'environ 6,5~\%\footnote{Cf. \ref{subsec:vit_cvg}}.

Cherchons à décomposer les différents effets en jeu ici. Dans notre modèle réduit du chapitre \ref{intuition} nous concluions que les émissions varient avec les droits de douane selon \ref{eq:eq_intuition_delta_em}\footnote{$\left({\partial E}\right)/\left( {\partial t} \right) = \left( {E_H^0 \eta_H \chi_F + E_F^0 \eta_F \mu_H} \right)/\left( {P_F^0 \mu_H - P_H^0 \chi_F} \right).$}, c'est-à-dire qu'elles augmentent avec l'introduction d'un droit de douane si les émissions du pays producteurs sont suffisamment basses par rapport à celles du pays mettant le droit de douane en place. Cherchons à montrer ce phénomène, en décomposant notre suppression de droits de douane.

\subsection{Suppression des droits de douane d'un pays pour un bien}\label{subsec:ae_anp}

Choisissons de retirer les droits de douane, d'un pays et d'un bien vers un autre pays moins émetteur sur ce bien (quantité $\times$ potentiel), idéalement ces droits de douane sont élevés. Nous choisissons l'Asie de l'Est comme région importatrice et les deux Amériques et l'Union européenne comme régions exportatrices, car toutes quatre sont respectivement importatrice et exportatrices nettes (cf. figure \ref{fig:sankey2}). On constate également que l'Asie de l'Est importe en valeur beaucoup de produits animaliers. Nous choisissons donc de faire une première décomposition, en retirant les droits de douane à l'import de produits issus de l'élevage en Asie de l'Est.

L'équilibre atteint après avoir changé les droits de douane conduit à une faible réduction de 0,32~\% des émissions à l'échelle mondiale. En regardant le détail par régions, on constate en effet que les émissions de notre région importatrice ont été réduites de 5,2~\%, celles de nos trois régions exportatrices ont elles similairement augmenté de 0,5-0,6~\%.

Concernant les productions, on constate\footnote{Cf. annexe \ref{annexe:ae_anp} pour tableau de résultats} une diminution en Asie de l'Est, des produits animaliers (tous de -5,3~\%\footnote{Les biens sont, dans notre modèle, tous issus du même processus, ils évoluent donc conjointement.}), ainsi que les biens utilisés pour l'alimentation animale (tourteaux -6,8~\%, fourrage -1,6~\%). En parallèle, on observe une réduction de l'import de biens servant (directement ou non) à l'alimentation animale, de 9~\% pour le soja, de 8~\% pour les tourteaux et de 5~\% pour le maïs. Cette réduction est de plus, plus importante pour les aliments venant des trois régions concernées par la suppression des droits de douane. On observe également une très forte réduction de l'import de produits animaliers venant des autres régions, allant pour les produits animaliers de 20~\% (pour la viande rouge) à 62-66~\% (pour le porc) (on observe même une réduction de près de moitié pour le sud-américain), et allant de 10 à 4~\% pour leur alimentation.

Parallèlement, on observe une diminution des prix à la consommation allant de 18~\% à 7~\% pour ces biens, et de moins d'un pourcent pour tous les autres biens. Mais étant donné que la consommation alimentaire totale ne varie que faiblement (+0,5~\%), on observe une légère diminution de la consommation des autres biens. À l'inverse, la consommation totale de produits agricoles diminue dans nos trois autres régions, et ce très légèrement, en parallèle d'une augmentation de moins d'un pourcent des prix à la consommation, des biens de la chaîne de valeur animale. Ces variations sont similaires pour les prix producteurs.

Finalement, on observe un comportement similaire à celui de notre modèle réduit, malgré l'influence du reste du monde qui n'est pas considéré dans notre mini-modèle.


\subsection{Suppression de droits de douane dans tous les pays sur un seul bien}

\subsubsection{Bien individuel non animal}

Ici nous supprimons alternativement les droits de douane pour un bien à la fois. Intuitivement deux choses peuvent se passer~: ou bien le bien est utilisé dans l'alimentation animale ou bien il ne l'est pas. S'il ne l'est pas, ce bien substituera d'autres aliments, dont les animaliers dans l'alimentation humaine, si au contraire il l'est suffisamment, la suppression des droits de douane conduira à une diminution du prix de l'alimentation animale et donc en bout de chaîne des produits animaliers, qui conduira à une augmentation de leur consommation. Si la consommation de produit animalier augmente, on s'attend à ce que les émissions totales augmentent. Parallèlement, pour un bien qui est utilisé majoritairement dans l'alimentation animale, uniquement dans une partie des pays, la suppression des droits de douane, risque d'augmenter le prix de l'alimentation animale uniquement dans ces pays, car les prix nationaux pour ce bien les pays l'exportant vont augmenter. Cet effet pourrait même aller jusqu'à réduire les émissions de GES dans certains pays. Le tableau \ref{tab:res_item} montre l'impact sur les émissions globales de la suppression, des droits de douane lui correspondant. Intuitivement, une autre chose peu se passer, si le bien n'est pas significativement utilisé dans la consommation animale, mais qu'il a un potentiel émissif supérieur à d'autres biens, l'augmentation des émissions due à celle de sa consommation risque de ne pas recouvrir la diminution des émissions du secteur animal suite à la baisse de sa consommation.

\begin{table}[hbt!]
    \centering
    \resizebox{\textwidth}{!}{%
        \begin{tabular}{l|ccccccccc}

                                                                         & MAI            & RIC\tnote{a}  & WHE           & CAK             & CER             & SOY             & V\&F             & R\&T            & POT              \\ \hline
            $\Delta$ des émissions totales                               & +1,4           & +0,41         & +0,19         & +0,127          & +0,047          & +0,045          & +0,036           & +0,025          & +0,024           \\
            $\Delta$ du prix de l'alimentation animale                   & {[}-21;+3,3{]} & {[}-2,3;+4{]} & {[}-3,8;+5{]} & {[}-1,7;+0,3{]} & {[}-1,2;+1,3{]} & {[}-4,9;+4,1{]} & {[}-0,7;+0,47{]} & {[}-1,1;+0,6{]} & {[}-0,5;+0,04{]} \\
            Nbr de régions avec $\partial E/\partial t >0$               & 5              & 11            & 8             & 7               & 9               & 4               & 8                & 12              & 6                \\
            Nbr de régions avec $\partial P_\text{feed}/ \partial t > 0$ & 11             & 12            & 10            & 8               & 9               & 12              & 6                & 9               & 3
        \end{tabular}%
    }
    \vspace{0.5em}
    \resizebox{\textwidth}{!}{%
        \begin{tabular}{l|cccc|ccccccc|ccc}
                                                                         & OSD             & SUG             & OCR     & OIL     & BAN     & CIT     & CTS          & PAK     & COT   & OPF   & CTL   & TOM               & SGC     & OILP    \\ \hline
            $\Delta$ des émissions totales                               & -0,111          & -0,09           & -0,017  & -0,008  & $0^-$   & $0^-$   & $0^-$        & $0^-$   & $0^-$ & $0^-$ & $0^-$ & $0^+$             & $0^+$   & $0^+$   \\
            $\Delta$ du prix de l'alimentation animale                   & {[}-1,5;+0,2{]} & {[}-0,3;+1,7{]} & $\sim$0 & $\sim$0 & $\sim$0 & $\sim$0 & {[}0;+0,3{]} & $\sim$0 & 0     & 0     & 0     & {[}-0,18;$0^+${]} & $\sim$0 & $\sim$0 \\
            Nbr de régions avec $\partial E/\partial t >0$               & 10              & 3               & 0       & 9       & 3       & 3       & 0            & 1       & 0     & 0     & 0     & 3                 & 4       & 6       \\
            Nbr de régions avec $\partial P_\text{feed}/ \partial t > 0$ & 11              & 11              & 9       & 4       & 8       & 4       & 13           & 1       & 0     & 0     & 0     & 1                 & 4       & 13
        \end{tabular}%
    }
    \begin{tablenotes}
        \item [a] Pour assurer la convergence de cette simulation, il a fallu d'abord libéraliser les importations n'allant pas en Asie de l'Est, puis réduire progressivement les droits de douane appliqués par la région.
        \item Les colonnes sont classées selon leur impact sur les émissions totales.
        \item Les valeurs telles quelles ne sont pas significatives, nous nous intéressons plutôt au signe.
    \end{tablenotes}
    \caption{Conséquence de la suppression des droits de douane liés à chaque bien.}
    \label{tab:res_item}
\end{table}

On observe plusieurs choses~:
\begin{itemize}
    \item globalement la suppression des droits de douane conduit à l'augmentation des émissions totales~;
    \item les biens qui sont utilisés presque exclusivement pour la consommation humaine (tomates, bananes, agrumes, huiles, sucres) ou dans une chaîne de valeur ne permettant pas l'alimentation animale (cultures sucrières, les différents états du coton, cœur de palmier), conduisent à une faible réduction totale des émissions~;
    \item la diminution des prix de l'alimentation animale dans des pays concorde avec l'augmentation dans d'autres, conduisant à une augmentation des émissions dans une partie uniquement des pays~;
    \item pour les biens majoritaires dans tous les pays pour l'alimentation animale (maïs, tourteaux et soja), on observe une certaine complémentarité entre les régions, ou le prix de l'alimentation animale est réduite et les émissions du pays augmente ou bien c'est l'inverse~;
    \item les biens servant aussi bien à l'alimentation animale qu'humaine, conduisent à une augmentation totale des émissions, tout en augmentant dans de nombreux pays le prix de l'alimentation animale.
\end{itemize}

Regardons plus en détail ce qui se passe lorsque l'on supprime les droits de douane sur le maïs, le riz. Parallèlement, regardons le cas des oléagineux et des autres cultures, qui semblent aller à l'envers des intuitions.

La libération des échanges de maïs, conduit à \textbf{blabla} augmentation des prix à la prod du riz (wtf).

Pour le riz, qui a comparativement un potentiel émissif plutôt élevé, on observe \textbf{blabla}



En retirant droits  les droits de douane de tous les pays sur les tourteaux et étant donné que ces derniers sont produits à partir d'oléagineux, on s'attendrait à observer une substitution dans les échanges entre les deux, en faveur des tourteaux. Finalement, le prix de la production de produits d'origine animale serait réduite, et on s'attendrait à une augmentation totale des émissions. Sauf que ce n'est pas ce qui se passe. \textbf{Rappeler le pourcentage de bouf dans le monde utilisé que pour ces satanées bestioles. 80 un truc ĉ ça, mais faire un graphe pour qu'on voit avec nos données.}



\subsubsection{Produits transformés d'origine animale}

Ici, nous opérons une libération totale des droits de douane sur les produits animaliers, ce qui correspond à une extension du cas vu dans la sous-section \ref{subsec:ae_anp}.

\subsubsection{Du produit animal le moins émetteur}

\textbf{Je crois qu'on va souligner un pb ici, vu que nos fonctions de prod, font qu'on produit en parallèle tous les ANP, on va finir par une augmentation conjointe de tous les ANP}


\subsection{Suppression de tous les droits de douane}

Comme on a pu le voir précédemment la suppression des droits de douane, quelqu'ils soient, semble conduire à une diminution du prix de l'alimentation animale ou bien directement des produits animaliers, ce qui conduit à une augmentation de leur consommation, donc à une augmentation des émissions totales.


\section{Sensibilité du modèle}

\subsection{Sensibilité aux paramètres}

\subsection{Sensibilité à la vitesse de convergence}\label{subsec:vit_cvg}

Pour faire converger le modèle vers une solution, nous opérons une suite d'équilibres intermédiaires représentant chacun un niveau de douane réduit. Nous ne pouvons, en effet actuellement pas atteindre l'équilibre fil correspondant à une absence de droits de douane directement à partir de l'équilibre initial, le solveur (\textit{mcp} ou \textit{cns} de GAMS\footnote{\textbf{Allez voir si y a pas des trucs écrits dans le cours de CG. Ou dans doc GAMS\dots}}) ne trouve pas de solution. Une première solution consiste à succéder une suite de chocs de -1~\% (ou de -0,5~\%) de droits de douane. Ce résultat nous conduit à une augmentation totale des émissions de 6,7~\%. À l'inverse, une évolution plus brusque de sept chocs de 10~\% jusqu'à -70~\%, puis de chocs de 4~\% jusqu'à la suppression totale des droits de douane conduit à une augmentation de 6,27~\%.

Cette observation témoigne en partie de l'importance du rythme d'application de politique dans leur impact. Ici, un changement plus brutal conduit à une moindre évolution de notre système. \textbf{Trouver des papiers qui parlent du rythme d'implantation, dans son impact.}

\textbf{Faire un graphe montrant en abscisse \% de droits de douane en -, et en ordonnée \% de GES en +, sauf que gams, et je sais pas comment stocker les valeurs intermédiaires d'une boucle. Demander à CG}
