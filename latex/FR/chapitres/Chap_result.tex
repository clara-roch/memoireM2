\section{Conséquence de la suppression des droits de douane}

Pour comprendre l'effet des droits de douane sur les émissions de GES du secteur, nous partons de la situation actuelle. Nous changeons ensuite de cet équilibre initial les valeurs des droits de douane pour les annuler, le second équilibre atteint, nous pouvons constater les émissions finales dans cette situation contrefactuelle et déduire l'impact des droits de douane.

La suppression des droits de douane dans notre modèle, avec le paramétrage présenté, augmente les émissions totales d'environ 6,1~\%.

Cherchons à décomposer les différents effets en jeu ici. Dans notre modèle réduit du chapitre \ref{intuition} nous concluions que les émissions variaient avec les droits de douane selon \ref{eq:eq_intuition_delta_em}\footnote{$\left({\partial E}\right)/\left( {\partial t} \right) = \left( {E_H^0 \eta_H \chi_F + E_F^0 \eta_F \mu_H} \right)/\left( {P_F^0 \mu_H - P_H^0 \chi_F} \right).$}, c'est-à-dire qu'elles augmentent avec l'introduction d'un droit de douane si les émissions du pays exportateurs sont suffisamment basses par rapport à celles du pays mettant le droit de douane en place. Cherchons à montrer ce phénomène, en décomposant notre suppression de droits de douane.

\subsection{Suppression des droits de douane d'une région pour un bien}\label{subsec:ae_anp}

Choisissons de retirer les droits de douane d'une région et d'un bien vers d'autres pays moins émettrice sur ce bien (quantité $\times$ potentiel), idéalement ces droits de douane sont élevés. Nous choisissons l'Asie de l'Est comme région importatrice et les deux Amériques et l'Union européenne comme régions exportatrices, car toutes quatre sont respectivement importatrice et exportatrices nettes (cf. figure \ref{fig:sankey2}). On constate également que l'Asie de l'Est importe en valeur beaucoup de produits animaliers\footnote{Cf. figure détaillée des échanges de produits animals \ref{fig:sankey_anp}.}. Nous choisissons donc de faire une première décomposition, en retirant les droits de douane à l'import de produits issus de l'élevage en Asie de l'Est.

L'équilibre atteint après avoir changé les droits de douane conduit à une faible réduction de 0,34~\% des émissions à l'échelle mondiale. En regardant le détail par régions, on constate en effet que les émissions de notre région importatrice ont été réduites de 5,6~\%, celles de nos trois régions exportatrices ont elles similairement augmenté de 0,6~\%.

Concernant les productions, on constate une diminution en Asie de l'Est, des produits animaliers (tous de -5,7~\%\footnote{Les biens sont, dans notre modèle, tous issus du même processus, leurs productions évoluent donc conjointement.}), ainsi que les biens utilisés pour l'alimentation animale (tourteaux -7,2~\%, fourrage -1,7~\%). En parallèle, on observe une réduction de l'import de biens servant (directement ou non) à l'alimentation animale, de 8,2~\% pour le soja, de 7,7~\% pour les tourteaux et de 5,5~\% pour le maïs. Cette réduction est, de plus, plus importante pour les aliments venant des trois régions concernées par la suppression des droits de douane. On observe également une très forte augmentation de l'import de produits animaliers venant des trois régions, allant pour de $\times$9 (pour les produits laitiers d'Amérique du Sud) à +10~\% (pour la volaille sud-américain), et même une réduction des importations de porc sud-américain de 44~\%, de la viande rouge sud-américaine de 19~\% et européenne de 3,5~\% (ces valeurs négatives s'expliquent par le processus commun de production de tous les produits d'origine animale).

Parallèlement, on observe en Asie de l'Est, une diminution des prix à la consommation allant jusqu'à 17~\% pour ces biens, et de moins d'un pourcent pour tous les autres biens. Mais étant donné que la consommation alimentaire totale ne varie que faiblement (+0,5~\%), on observe une légère diminution de la consommation des autres biens. À l'inverse, la consommation totale de produits agricoles diminue dans nos trois autres régions, et ce, très légèrement, en parallèle d'une augmentation de moins d'un pourcent des prix à la consommation, des biens de la chaîne de valeur animale. Ces variations sont similaires pour les prix producteurs.

Finalement, on observe un comportement similaire à celui de notre modèle réduit, malgré l'influence du reste du monde qui n'était pas considéré dans notre mini-modèle.


\subsection{Suppression de droits de douane dans tous les pays sur un seul ou groupe de biens}

\subsubsection{Bien individuel non animal}

Ici nous supprimons alternativement les droits de douane pour un bien à la fois, afin de se donner une intuition de ce qui se passe lorsque l'on retire tous les droits de douane à la fois.

Intuitivement deux choses peuvent se passer~: ou bien le bien est utilisé dans l'alimentation animale ou bien il ne l'est pas. S'il ne l'est pas, ce bien substituera d'autres aliments, dont ceux d'origine animale dans l'alimentation humaine, si au contraire il l'est suffisamment, la suppression des droits de douane conduira à une diminution du prix de l'alimentation animale et donc en bout de chaîne des produits animaliers, qui conduira à une augmentation de leur consommation. Si la consommation de produit animalier augmente, on s'attend à ce que les émissions totales augmentent. Parallèlement, pour un bien qui est utilisé majoritairement dans l'alimentation animale, uniquement dans une partie des pays, la suppression des droits de douane, risque d'augmenter le prix de l'alimentation animale uniquement dans ces pays, car les prix nationaux pour ce bien les pays l'exportant vont augmenter. Cet effet pourrait même aller jusqu'à réduire les émissions de GES dans certains pays.

Le tableau \ref{tab:res_item} montre l'impact sur les émissions globales de la suppression, des droits de douane lui correspondant. Intuitivement, une autre chose peu se passer, si le bien n'est pas significativement utilisé dans la consommation animale, mais qu'il a un potentiel émissif supérieur à d'autres biens, l'augmentation des émissions due à celle de sa consommation risque de ne pas recouvrir la diminution des émissions du secteur animal suite à la baisse de sa consommation.

\begin{table}[hbt!]
    \centering
    \caption{Conséquence de la suppression des droits de douane liés à chaque bien (transposée et ordonnée).}
    \label{tab:res_item}
    \begin{tabularx}{\textwidth}{>{\raggedleft\arraybackslash}b{1.1in}|*{5}{>{\centering\arraybackslash}X}}
        \hline
        \textbf{Bien}      & \textbf{$\Delta$ des émissions totales} & \textbf{$\Delta$ de la valeur moyenne de $P_\text{feed}$} & \textbf{Espace des $\Delta P_\text{feed}$} & \textbf{Nbr de régions avec $\Delta E >0$} & \textbf{Nbr de régions avec $\Delta P_\text{feed} > 0$} \\
        \hline
        Maïs               & +1,4                                    & -0,05                                                     & {[}-21;+3,3{]}                             & 5                                          & 11                                                      \\
        Riz                & +0,41                                   & +0,9                                                      & {[}-2,3;+4{]}                              & 11                                         & 12                                                      \\
        Blé                & +0,19                                   & +0,7                                                      & {[}-3,8;+5{]}                              & 8                                          & 10                                                      \\
        Tourteaux          & +0,127                                  & -0,04                                                     & {[}-1,7;+0,3{]}                            & 7                                          & 8                                                       \\
        Autres céréales    & +0,047                                  & +0,01                                                     & {[}-1,2;+1,3{]}                            & 9                                          & 9                                                       \\
        Soja               & +0,045                                  & +0,18                                                     & {[}-4,9;+4,1{]}                            & 4                                          & 12                                                      \\
        Fruits et légumes  & +0,036                                  & -0,03                                                     & {[}-0,7;+0,47{]}                           & 8                                          & 6                                                       \\
        Racines et tuberc. & +0,025                                  & +0,08                                                     & {[}-1,1;+0,6{]}                            & 12                                         & 9                                                       \\
        Pommes de terres   & +0,024                                  & -0,03                                                     & {[}-0,5;+0,04{]}                           & 6                                          & 3                                                       \\
        Tomates            & $0^+$                                   & -0,02                                                     & {[}-0,18;$0^+${]}                          & 3                                          & 1                                                       \\
        Cultures sucrières & $0^+$                                   & 0                                                         & $\sim$0                                    & 4                                          & 4                                                       \\
        Huile de palme     & $0^+$                                   & $0^+$                                                     & $\sim$0                                    & 6                                          & 13                                                      \\
        Bananes            & $0^-$                                   & $0^+$                                                     & $\sim$0                                    & 3                                          & 8                                                       \\
        Agrumes            & $0^-$                                   & $0^-$                                                     & $\sim$0                                    & 3                                          & 4                                                       \\
        Graines de coton   & $0^-$                                   & $0^+$                                                     & {[}0;+0,3{]}                               & 0                                          & 13                                                      \\
        Cœur de palmiers   & $0^-$                                   & $0^+$                                                     & $\sim$0                                    & 1                                          & 1                                                       \\
        Coton              & $0^-$                                   & 0                                                         & 0                                          & 0                                          & 0                                                       \\
        Fruits de palmiers & $0^-$                                   & 0                                                         & 0                                          & 0                                          & 0                                                       \\
        Peluches de coton  & $0^-$                                   & 0                                                         & 0                                          & 0                                          & 0                                                       \\
        Huiles végétales   & -0,008                                  & +0,01                                                     & $\sim$0                                    & 9                                          & 4                                                       \\
        Autres cultures    & -0,017                                  & +0,01                                                     & $\sim$0                                    & 0                                          & 9                                                       \\
        Sucres             & -0,09                                   & +0,09                                                     & {[}-0,3;+1,7{]}                            & 3                                          & 11                                                      \\
        Oléagineux         & -0,111                                  & +0,15                                                     & {[}-1,5;+0,2{]}                            & 10                                         & 11                                                      \\
        \hline
    \end{tabularx}%

    \begin{tablenotes}
        \item Les lignes sont classées par ordre décroissant selon leur impact sur les émissions totales.
    \end{tablenotes}
\end{table}

On observe plusieurs choses~:
\begin{itemize}
    \item globalement la suppression des droits de douane conduit à l'augmentation des émissions totales~;
    \item les biens qui sont utilisés presque exclusivement pour la consommation humaine (tomates, bananes, agrumes, huiles, sucres) ou dans une chaîne de valeur ne permettant pas l'alimentation animale (cultures sucrières, les différents états du coton, cœur de palmier), conduisent à une faible réduction totale des émissions~;
    \item la diminution des prix de l'alimentation animale dans des pays concorde avec l'augmentation dans d'autres, conduisant à une augmentation des émissions dans une partie uniquement des pays~;
    \item pour les biens majoritaires dans tous les pays pour l'alimentation animale (maïs, tourteaux et soja), on observe une certaine complémentarité entre les régions, ou le prix de l'alimentation animale est réduite et les émissions du pays augmente ou bien c'est l'inverse~;
    \item les biens servant aussi bien à l'alimentation animale qu'humaine, conduisent à une augmentation totale des émissions, tout en augmentant dans de nombreux pays le prix de l'alimentation animale.
\end{itemize}

Regardons plus en détail ce qui se passe lorsque l'on supprime les droits de douane sur le maïs, le riz. Parallèlement, regardons le cas des oléagineux, qui semblent aller à l'envers des intuitions. L'annexe \ref{annexe:sankey} représente des graphes de Sankey des commerces extérieurs des différents biens que nous analysons ci-dessous.

La libération des échanges de maïs entraîne une diminution prix à la consommation du maïs, en Asie de l'Est, Moyen-Orient, Asie du SE, Chine et Afrique Sud Saharienne, cependant cette diminution implique une baisse du coût moyen de la nourriture animale uniquement en Asie de l'Est, Moyen-Orient et Chine~; on remarque que les deux régions pour lesquelles la baisse du prix du maïs n'entraîne pas une baisse du prix de l'alimentation animale sont des régions pour lesquelles l'augmentation du reste des prix de l'alimentation animale ne permet pas de baisser le panier moyen, en effet ces deux régions consomment désormais plus de céréales, de riz et des tubercules (parmi d'autres), hors les prix de ces denrées ont également augmenté. Concernant les autres régions, tous les prix constituant le panier d'aliments pour l'élevage, il est tout naturel que le prix de l'alimentation augmente. Cela conduit à une diminution du prix moyen des bien agricole uniquement pour l'Asie de l'Est (-4\%), le Moyen-Orient et la Chine ($~0^-$), la plus grande augmentation est ressentie en Amérique du Sud (+0,8~\%). Cependant, seule les cinq régions qui ont ressentie une baisse des prix à la consommation des prix du maïs observent une augmentation totale des émissions~; les variations d'émissions s’étalent de -1,6~\% pour l'Amérique du Sud, jusqu'à +2,1~\% pour l'Asie de l'Est. A l'échelle monde, l'augmentation des émissions est surtout allouée aux maïs (+3,6~\%) dont la demande en fertilisant augmente dans toutes les régions ayant vu le prix du maïs augmenter (jusqu'à +50~\% dans en Russie et Communautés des États indépendants, ou encore +22~\% en Amérique du Sud). Les émissions allouées aux cheptels augmentent également, mais ne sont pas dues à une augmentation de l'usage d'intrants sur les pâtures (qui augmentent au maximum de 1~\% en AmS), mais bien à l'augmentation de la production qui augmentent dans nos cinq régions (plus également légèrement l'Océanie), jusqu'à +22~\% en Asie de l'Est. À l'opposé la réduction maximale de production de viande n'est que de 2~\% en Amérique du Sud.

Pour le riz, qui a comparativement un potentiel émissif plutôt élevé, on observe une diminution des prix à la consommation très forte en Asie de l'Est (-63~\%), puis plus modérée de 8 à 2~\% en Amérique centrale, dans toute l'Europe et la Chine, les prix augmentent de plus de 10~\% en Océanie, Asie du Sud et du SE et Amérique du Nord. L'Asie du Sud et du SE sont les deux plus grandes régions exportatrices de riz. L'Océanie et l'Amérique du Nord, sont quant à elles deux régions ne consommant pas beaucoup de riz, cette augmentation n'est donc pas significative. Pour contrecarrer, ces augmentations de prix à la consommation, toutes les régions à l'exception de l'Asie de l'Est (-22~\%) et de l'Amérique centrale (-4~\%) augmentent la surface allouée à la culture du riz, de +5~\% en Asie du SE, et +15~\% en Asie du Sud, à +70~\% et +207~\% en Amérique du Nord et Océanie. Parallèlement on observe une augmentation allant dans le même sens, mais un peu plus forte de l'utilisation des intrants (i.e. augmentation de l'usage de fertilisants pour une même surface cultivée). Étant donné que les émissions du riz sont avant tout lié à l'usage du sol et de fertilisants (cf. \ref{data}), on explique facilement l'augmentation de +20~\% des émissions de GES lié au riz dans le monde. Parallèlement, on observe une légère augmentation des émissions de GES (de moins de 1~\%) des substituts au riz dans l'alimentation humaine (blé, pommes de terre, céréales, etc.), notamment due à l'augmentation de l'usage d'intrants, permettant de palier à la perte de terre en faveur du riz, mais aussi une diminution des émissions de l'élevage. En effet, on observe une légère augmentation de moins d'un pourcent, des prix de l'alimentation animale dans toutes les régions, sauf en Asie de l'Est (-3~\%), en Amérique centrale ($0^-$), en Asie du Sud et SE (+4~\%), qui entraîne une diminution de la production animale, allant jusqu'à 4~\% en Asie du SE, et donc des émissions qui leur sont allouées.

En retirant les droits de douane de toutes les régions sur les oléagineux et étant donné qu'ils permettent la production de tourteaux, on s'attendrait intuitivement à observer une substitution dans les échanges entre les deux, en faveur des oléagineux. De plus, notons que les tourteaux sont produits avec nos biens à partir de graines de coton, d'oléagineux et de cœurs de palmier~; et étant donné la libéralisation des prix oléagineux. L'Amérique du Nord, qui est la plus grande région exportatrice d'oléagineux et également la seconde pour les tourteaux, comme les prix d'importation ont été réduits, on observe une augmentation de 10~\% de leur import depuis l'Amérique du Sud et l'UE, au total le prix à la production des tourteaux y diminue de 0,2~\%. Comme l'UE, n'exporte pas de tourteaux, mais uniquement des oléagineux, Les prix à la consommation des oléagineux augmentent, mais les prix des tourteaux diminuent, dû à la diminution du prix de production en Amérique du Nord (qui leur importe la moitié de leurs tourteaux), en Amérique du Sud à l'inverse (plus grande région exportatrice de tourteaux, mais non-exportatrice d'oléagineux) les prix à la consommation des oléagineux et des tourteaux diminuent. On observe ensuite une réduction des prix à la consommation des tourteaux s'accompagnant d'une augmentation de leur utilisation dans l'alimentation animale, d'une diminution du coût de celle-ci, et donc d'une augmentation de la production animale et des émissions totales de la région pour deux tiers des régions. Cependant, la plus grande baisse de production en Asie de l'Est (-1,6~\% d'émissions) couvrent les la faible augmentation dans les autres régions. Finalement, l'effet de la libéralisation des oléagineux est différent de celle des tourteaux.


\subsubsection{De l'ensemble des biens non-animalier}\label{sousec_all_non_kl}

Si on somme, l'ensemble des effets individuels de chacune des politiques de la section précédente, on obtient +2,1~\% d'émissions en plus. Cependant, le retrait de l'ensemble des droits de douane conduit en réalité à une augmentation plus importante des émissions de +3,3~\%.


\subsubsection{Produits transformés d'origine animale}

Ici, nous opérons une libération totale des droits de douane sur les produits animaliers, ce qui correspond à une extension du cas vu dans la sous-section \ref{subsec:ae_anp}. Contrairement au cas Asie de l'Est, on observe une augmentation émissions totales de 3,4~\%, on observe une diminution des émissions dans les régions Europe hors-UE, Asie de l'Est, Amérique centrale, toute l'Afrique et la Chine, ces régions sont toutes six des régions importatrices de produits finis d'origine animale, et diminuent de ce fait leur production nationale, leur demande de travail et d'intrants pour l'élevage. Le prix de l'alimentation animale y varie de -4 à +1,5~\%, mais la production y diminue bien (de -27 à -3~\%). Parallèlement, étant donné que ces régions consomment plus de viande, lait et œufs (jusqu'à +14~\% pour la viande rouge en Europe hors-UE), elle en importe plus (+34~\% pour Europe hors-UE), et importent moins de biens à destination de l'alimentation animale (-30~\% toujours en nUE). Ces réductions sont accompagnées d'une augmentation de l'élevage dans toutes les autres régions (jusqu'à +67~\% en Océanie, ou +9 et +7~\% dans l'UE et en Amérique du Sud, deux grandes régions exportatrices de produits d'origine animale). Leurs imports de tourteaux augmentent, et ce, principalement depuis les six régions qui bénéficient de cette libéralisation. On observe également, que les régions exportatrices de viande et autres, n'augmentent ni la production pour les plantes fourragères, ni l'usage d'intrants. On observe donc une re-répartition des spécialisations, les régions exportatrices de viande se focalisent sur leur production, en important des régions importatrice de viande les biens servant à l'alimentation animale. Finalement, on observe également de faibles diminutions des émissions liées à l'ensemble des biens majoritairement utilisées pour l'alimentation humaine, comme les huiles, les fruits et légumes, ces diminutions témoignent de leur remplacement dans l'alimentation humaine en défaveur des produits d'origine animale.

\subsubsection{D'un seul produit issu de l'élevage relativement peu émetteur}

Nous supprimons d'abord l'ensemble les droits de douane des produits laitiers, ils sont consommés en quantités importantes dans toutes les régions. L'augmentation résultante est moins importante qu'après la suppression des droits de douane de tout le secteur, mais reste significativement importante (+2,2~\%). Pour ne pas tout re-détailler, nous observons dans les grosses lignes les mêmes effets que lors de la suppression sur tous les produits animaliers. Mais surtout, on observe du fait de notre choix de modélisation des produits issus de l'élevage, que la production augmente de la même manière pour l'ensemble des produits du secteur, chacun étant produit conjointement par une fonction de production Léontief. Ainsi, pour compenser l'augmentation forcée des autres prix animaliers, on observe une diminution des prix à la consommation des autres denrées issus de l'élevage.

À l'inverse lorsque nous retirons les droits de douane sur les œufs, nous observons une réduction de 0,2~\% des émissions de GES. Les réductions, des émissions sont dues cette fois-ci à une réduction plus importante de la production dans certaines régions d'une augmentation d'en d'autres. Les prix des œufs ayant diminué, la consommation est transférée d'autre produits animaliers, vers ces premiers et demande donc une plus faible production pour les autres produits animaliers. Ainsi, malgré la fonction de production conjointe entre les différents produits issus de l'élevage, il est quand même possible d'observer une diminution, quoique faible.


\subsection{Suppression de tous les droits de douane}

Comme on a pu le voir précédemment la suppression des droits de douane, quelqu'ils soient, semble conduire à une diminution du prix de l'alimentation animale ou bien directement des produits animaliers, ce qui conduit à une augmentation de leur consommation, donc à une augmentation des émissions totales. Similairement à ce que nous avons pu noter au sein des produits non-animalier dans la section \ref{sousec_all_non_kl}, les effets de la suppression des droits de douane de tous les secteurs hors élevage, avec le secteur élevage, ne s'additionner pas parfaitement. La somme de leurs effets est de +6,7~\%, tandis que l'effet d'une libéralisation totale implique une augmentation — légèrement inférieure — de +6,1~\%.

Les résultats montrent une augmentation du prix moyen de l'alimentation animale supérieur à celle observée lorsque seuls les produits d'origine animale voient leur droit de douane retiré, on constate de plus une variation plus grande. Le prix moyen augmente de 5~\% (contre 2,6 dans l'autre contrefactuel), avec une augmentation allant jusqu'à 33~\% en Océanie (contre +29). Ce retrait total des droits de douane conduit à une augmentation des consommations finales des différents biens allant jusqu'à +5,7~\% pour le soja, à - 7,8~\% pour les cultures sucrières. Mais surtout on observe que la quantité totale de produits d'origine animale diminue de 1,8~\%, alors même que les émissions du secteur augmentent pour tous les biens (de +1,6~\% pour les produits laitiers à +9~\% pour les volailles). Les émissions du secteur sont portées majoritairement par l'Océanie (+63~\%), l'Asie de l'Est (+47~\%), l'Asie du SE (+20~\%) et le Moyen-Orient (+10~\%), cette augmentation est due à une augmentation des productions des mêmes pourcentages. À l'inverse toutes les autres régions voient leur production d'élevage et leurs émissions associées diminuer. Ce faisant les émissions du secteur s'équilibrent. Au total, la région qui voit le plus ces émissions diminuer dans ce scénario est la Chine, avec - 2,7~\%, et cette baisse n'est pas due à un déplacement des émissions vers l'import, mais bien à une réduction de la consommation de 1,2~\% et à de forte émissions initiales.


\section{Sensibilité du modèle aux paramètres}

Dans cette sous-section, nous analysons l'impact du choix des paramètres de comportement sur les conséquences en terme d'émissions de GES dans nos différents scenarii contrefactuels, en faisant varier leur valeur un à un. Le tableau \ref{tab:sensibilite_param} expose la variation d'émissions de GES dans quatre scenarii, avec les changements de paramètres choisis.

\begin{table}[hbt!]
    \centering
    \caption{Influence de la modification des paramètres sur les émissions finales des différents scenarii.}
    \label{tab:sensibilite_param}
    \begin{tabular}{c|cccc}
        \hline
        Paramètre changé           & Maïs & Tout hors-élevage & Élevage & Libéralisation totale \\ \hline
        $\emptyset$\tnote{a}       & +1,4 & +3,3              & +3,4    & +6,1                  \\
        $\kappa = 0.1$             & +0,9 & +3,4              & +1,8    & +4,9                  \\
        $\kappa = 2$               & +1,6 & +1,8              & +5,8    & +6,4                  \\
        $\kappa_\text{feed} = 0.1$ & +0,8 & +1,9              & +2,3    & +4,3                  \\
        $\kappa_\text{feed} = 2$   & +1,8 & +3,9              & +3,1    & +7,1                  \\
        $\sigma^k = 10$            & +2,4 & +5                & +7      & +13                   \\
        $\varsigma = 1$            & +1,6 & +4,2              & +3,1    & +7,3                  \\
        \hline
    \end{tabular}%
    \begin{tablenotes}
        \footnotesize
        \item[a] Les paramètres initiaux étaient~: $\kappa$ = 0,6, $\kappa_\text{feed}$ = 0,9, $\sigma^k \in$ [2,6, 10,1] et $\varsigma$ = 0,25.
    \end{tablenotes}
\end{table}

Nous constatons d'abord que le changement de paramétrage, ne change pas la nature des résultats~: le retrait de droits de douane entraîne l'augmentation des émissions de GES. Comme on pouvait s'y attendre, lorsque les élasticités de substitution $\kappa$ et $\kappa_\text{feed}$ diminuent la variation absolue des émissions est réduite, puisque plus elles sont faibles moins les consommateur$\cdot$ice$\cdot$s / éleveur$\cdot$euse$\cdot$s voudront changer leur panier de biens, ce qui conduit à réduire les variations de productions. Ensuite, les élasticités d'Armington $\sigma^k$ lorsqu'elle augmente témoigne d'une plus grande indifférence entre la provenance des biens, son augmentation conduit donc à réaction plus importante aux politiques commerciales mondiales.