\section{Conséquence de la suppression des droits de douane}

Pour comprendre l'effet des droits de douane sur les émissions de GES du secteur, nous partons de la situation actuelle. Nous changeons ensuite de cet équilibre initial les valeurs des droits de douane pour les annuler, le second équilibre atteint, nous pouvons constater les émissions files dans cette situation contrefactuelle et déduire de l'impacte des droits de douane.

La suppression des droits de douane dans notre modèle, avec le paramétrage présenté, augmente les droits de douane de l'ensemble du secteur d'environ 6,5~\%\footnote{cf. \ref{subsec:vit_cvg}}.

Cherchons à décomposer les différents effets en jeu ici. Dans notre modèle réduit du chapitre \ref{intuition} nous concluions que les émissions varient avec les droits de douane selon \ref{eq:eq_intuition_delta_em}\footnote{$\left({\partial E}\right)/\left( {\partial t} \right) = \left( {E_H^0 \eta_H \chi_F + E_F^0 \eta_F \mu_H} \right)/\left( {P_F^0 \mu_H - P_H^0 \chi_F} \right).$}, c'est-à-dire qu'elles augmentent avec l'introduction d'un droit de douane si les émissions du pays producteurs sont suffisamment basses par rapport à celles du pays mettant le droit de douane en place. Cherchons à montrer ce phénomène, en décomposant notre suppression de droits de douane.

\subsection{Suppression des droits de douane d'un pays pour un bien}

Choisissons de retirer les droits de douane, d'un pays et d'un bien vers un autre pays moins émetteur sur ce bien (quantité $\times$ potentiel), idéalement ces droits de douane sont élevés. Nous choisissons l'Asie de l'Est comme région importatrice et les deux Amériques et l'Union européenne comme régions exportatrices, car toutes quatre sont respectivement importatrice et exportatrices nettes (cf. figure \ref{fig:sankey2}). On constate également que l'Asie de l'Est importe en valeur beaucoup de produits animaliers. Nous choisissons donc de faire une première décomposition, en retirant les droits de douane à l'import de produits issus de l'élevage en Asie de l'Est.

L'équilibre atteint après avoir changé les droits de douane conduit à une faible réduction de 0,32~\% des émissions à l'échelle mondiale. En regardant le détail par régions, on constate en effet que les émissions de notre région importatrice ont été réduites de 5,2~\%, celles de nos trois régions exportatrices ont elles similairement augmenté de 0,5-0,6~\%.

Concernant les productions, on constate\footnote{cf. annexe \ref{annexe:ae_anp} pour tableau de résultats} une diminution en Asie de l'Est, des produits animaliers (tous de -5,3~\%\footnote{Les biens sont, dans notre modèle, tous issus du même processus, ils évoluent donc conjointement.}), ainsi que les biens utilisés pour l'alimentation animale (tourteaux -6,8~\%, fourrage -1,6~\%). En parallèle, on observe une réduction de l'import de biens servant (directement ou non) à l'alimentation animale, de 9~\% pour le soja, de 8~\% pour les tourteaux et de 5~\% pour le maïs. Cette réduction est de plus, plus importante pour les aliments venant des trois régions concernées par la suppression des droits de douane. On observe également une très forte réduction de l'import de produits animaliers venant des autres régions, allant pour les produits animaliers de 20~\% (pour la viande rouge) à 62-66~\% (pour le porc) (on observe même une réduction de près de moitié pour le sud-américain), et allant de 10 à 4~\% pour leur alimentation.

Parallèlement, on observe une diminution des prix à la consommation allant de 18~\% à 7~\% pour ces biens, et de moins d'un pourcent pour tous les autres biens. Mais étant donné que la consommation alimentaire totale ne varie que faiblement (+0,5~\%), on observe une légère diminution de la consommation des autres biens. À l'inverse, la consommation totale de produits agricoles diminue dans nos trois autres régions, et ce très légèrement, en parallèle d'une augmentation de moins d'un pourcent des prix à la consommation, des biens de la chaîne de valeur animale. Ces variations sont similaires pour les prix producteurs.


\section{Sensibilité du modèle}

\subsection{Sensibilité aux paramètres}

\subsection{Sensibilité à la vitesse de convergence}\label{subsec:vit_cvg}

Pour faire converger le modèle vers une solution, nous opérons une suite d'équilibres intermédiaires représentant chacun un niveau de douane réduit. Nous ne pouvons, en effet actuellement pas atteindre l'équilibre fil correspondant à une absence de droits de douane directement à partir de l'équilibre initial, le solveur (\textit{mcp} ou \textit{cns} de GAMS\footnote{\textbf{Allez voir si y a pas des trucs écrits dans le cours de CG. Ou dans doc GAMS\dots}}) ne trouve pas de solution. Une première solution consiste à succéder une suite de chocs de -1~\% (ou de -0,5~\%) de droits de douane. Ce résultat nous conduit à une augmentation totale des émissions de 6,7~\%. À l'inverse, une évolution plus brusque de sept chocs de 10~\% jusqu'à -70~\%, puis de chocs de 4~\% jusqu'à la suppression totale des droits de douane conduit à une augmentation de 6,27~\%.

Cette observation témoigne en partie de l'importance du rythme d'application de politique dans leur impact. Ici, un changement plus brutal conduit à une moindre évolution de notre système. \textbf{Trouver des papiers qui parlent du rythme d'implantation, dans son impact.}

\textbf{Faire un graphe montrant en abscisse \% de droits de douane en -, et en ordonnée \% de GES en +, sauf que gams, et je sais pas comment stocker les valeurs intermédiaires d'une boucle. Demander à CG}
