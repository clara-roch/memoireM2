L’agriculture, responsable d’environ un quart des émissions mondiales de gaz à effet de serre (GES), est influencée par les politiques commerciales, qui modifient les échanges, la spécialisation et la localisation des activités. Les droits de douane, en protégeant certaines filières ou en limitant les importations, ont des effets indirects et parfois contre-intuitifs sur les émissions, en altérant les équilibres de marché et les choix de production et de consommation. Ce mémoire présente un modèle de commerce agricole en équilibre partiel, prenant en compte les consommations intermédiaires, l'usage d'intrants intensifs, mais pas la variation des surfaces agricoles totales. Les données utilisées proviennent du projet FABIO pour les flux de production et de commerce, FAOSTAT pour les prix, MAcMap pour les droits de douane, et GTAP pour les émissions et l’usage d’intrants. Le modèle est calibré sur l’année 2017 et couvre 14 régions et 30 biens agricoles, permettant une représentation fine des interactions entre les différents acteurs et produits. La simulation d'un monde sans droits de douane agricoles montrent une augmentation d’environ 6,1~\% des émissions totales de GES, du secteur. Cette hausse s’explique principalement par une baisse des coûts de l’alimentation animale et des produits d’élevage, stimulant leur consommation et leur production, qui sont parmi les plus émetteurs. Une analyse désagrégée révèle des effets contrastés selon les régions et les produits : les pays importateurs nets de viande voient leurs émissions locales diminuer, tandis que les pays exportateurs connaissent une hausse de leurs émissions, reflétant une spécialisation accrue dans la production animale. La suppression des droits de douane sur des produits végétaux comme le riz conduit à des résultats plus nuancés, avec des réductions marginales des émissions dans certains cas, notamment lorsque ces produits se substituent à des aliments plus émetteurs dans l’alimentation humaine ou animale. En conclusion, les droits de douane apparaissent comme un levier ambivalent pour la réduction des émissions agricoles, dont les effets dépendent étroitement des contextes régionaux et sectoriels. Leur suppression, bien que souvent présentée comme bénéfique pour l’efficacité économique, pourrait, sans mesures complémentaires, aggraver la pression climatique du secteur agricole.

\textbf{Mots clés~:} commerce international, agriculture, changement climatique, droits de douane, modélisation en équilibre partiel
