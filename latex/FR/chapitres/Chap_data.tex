\section{Données et traitement}

Le modèle est calibré sur des données de 2017, dernière année disponible dans les données GTAP.


\subsection{Quantités et valeurs}
Les données sont disponibles à l'échelle \textit{cbs}, qui correspond à une agrégation partielle, qui est celle de la colonne de gauche de l'annexe (??).

\subsection{Prix}
Nous avons utilisé les données de prix issus des bases FAOSTAT sur les prix producteur et sur les quantités totales et en valeurs échangées, pour chaque culture et chaque pays en désagrégé. Les prix donnés par FAOSTAT sont à un niveau de désagrégation inférieur à celui des données disponibles avec FABIO, pour simplifier l'exposé nous nommerons ce niveau \textit{item}. Une table de passage nous permet de passer de ces ?? \textit{item} à ?? \textit{cbs}.

L'utilisation de ces deux bases permet de recouvrir le plus de prix possible. Afin de s'assurer de la cohérence des prix, nous utilisons les données de production totale en quantité et en valeur des \textit{items}, agrégées au niveau monde, ainsi nous récupérons un prix moyen pour chaque culture correspondant à la valeur totale produite sur la quantité totale produite, et nous assurons que les prix issus des données producteurs et des données de commerce, dans chaque pays ne sont pas trop éloignés de cette valeur. La procédure de conservation des prix est la suivante~:
\begin{itemize}
    \item utiliser les {80\,\%} des prix d'\textit{item}, les plus proches du prix moyen mondial trouvé~;
    \item calculer la moyenne et l'écart type de la répartition de ces prix~;
    \item calculer pour l'ensemble des prix (pas uniquement les 80\,\% des prix les plus proches), l'écart à la moyenne en nombre d’écart-type~;
    \item calculer pour chaque pays l'écart-type moyen pour chaque pays, afin de déterminer si le pays a des prix habituellement élevés par rapport au prix mondiaux~;
    \item ne conserver que les prix ne s'éloignant pas de plus de deux écart-types en plus de leur écart moyen, calculé à l'étape précédente, des prix mondiaux.
\end{itemize}
On applique également ce tri, sur les prix agrégés au niveau supérieur des \textit{cbs}. Ensuite, on agrège les prix restant au niveau des \textit{cbs}.

% [expliquer ces affaires de exports imports]


\section{Paramètres de comportement}
Nous faisons les mêmes choix sur le paramétrage que dans \cite{Gouel2025}. Le tableau \label{tab:ela} résume les valeurs choisies pour les élasticités, ainsi que l'origine de ces choix.

\begin{table}[h]
    \centering
    \begin{tabularx}{\textwidth}{l >{\raggedright\arraybackslash}X >{\raggedright\arraybackslash}X}
        \textbf{Élasticité}        & \textbf{Description}                                          & \textbf{Origine}                                   \\ \hline
        $\epsilon = 0.5$           & opposé de prix de la demande pour le panier de bens agricoles & \cite{Comin2021}                                   \\
        $\kappa = 0.6$             & de substitution à la consommation                             & Valeur usuelle dans la littérature \cite{Rude2000} \\
        $\kappa_\text{feed} = 0.9$ & de substitution dans l'alimentation animale                   & \cite{Rude2000}                                    \\
        $\varsigma = 0.25$         & prix rendement des cultures                                   & \cite{Keeney2009}                                  \\
        $\sigma_k \in [2.6, 10]$   & de substitution d'Armington                                   & GTAP  \cite{Aguiar2022}                            \\
        \hline
    \end{tabularx}
    \caption{Origine des paramètres}
    \label{tab:ela}
\end{table}

Nous pouvons remarquer que l'élasticité de substitution entre les biens pour l'alimentation animal est plus élevée que celle pour l'alimentation humaine, cela témoigne deux choses, d'abord que l'alimentation animale est constitués d'un panel d'aliments moins diversifiés (les animaux mangent principalement des céréales et des tourteaux), ensuite que les humains veulent manger tel ou tel aliment, et non pas que pour son apport calorique.

Le choix de l'élasticité $\varsigma$ vient d'une moyenne des valeurs de l'étude GTAP \cite{Miller2009} qui porte sur les bio-carburants, idéalement la valeur aurait dû être différenciée entre les différentes cultures dans les différents pays pour exprimer les différentes rigidités dans chaque pays.

Les valeurs des élasticités de substitutions d'Armington, viennent quant à elles de la base de données GTAP \footcite{Aguiar2022}.