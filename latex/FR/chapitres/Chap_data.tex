\section{Données et traitements}

Le modèle est calibré sur des données de 2017, dernière année disponible dans les données GTAP. Comme souligné dans \cite{Gouel2021} il est important qu'un modèle recouvre le plus de biens agricoles, afin que l'élasticité $\varepsilon$ correspondent réellement à l'inverse de l'élasticité de la demande pour les biens agricoles, qui est estimé dans des papiers comme \cite{Comin2021} et non pas juste à une partie d'entre eux, nous avons donc choisi de représenter l'ensemble (ou presque) des éléments biens disponibles dans FABIO, en les agrégeant en trente-et-un groupes, qui représentent des éléments uniques ou bien des ensembles, les biens uniques, ont été choisi de telle sorte qu'ils représentent chacun plus d'un pourcent de la production mondiale totale\footnote{cf. table 2 de \cite{Gouel2021}}, les autres sont regroupés par catégories. Quant aux pays, nous avons choisi de les rassembler en quatorze régions, correspondant au regroupement des pays par sous-continents, en évitant d'avoir une région reste du monde, qui malgré qu'elle apparaisse dans les données FABIO\footnote{Modèle basé sur les données de FAOSTAT, représentant les échanges entre pays, les productions et les usages finaux, regroupant 186 pays et 127 biens agricoles, recouvrant ?? de la production mondiale \cite{Bruckner2019}} et GTAP\footnote{Base de données GTAP, représente l'économie mondiale, en se basant sur des données de comptabilité nationale, de commerce, de production, etc., répartie en vingt-et-une catégories de biens, et en 160 régions.}, n'avait pas de sens géographique ou politique\footnote{L'annexe \ref{annexe:reg} donne l'attribution des pays aux régions.}.


\subsection{Prix - FAOSTAT -- \textbf{à reprendre}}\label{subsec:prix}

Les données sont disponibles à l'échelle \textit{cbs}, qui correspond à une agrégation partielle, qui est celle de la colonne de gauche de l'annexe (??).

Nous avons utilisé les données de prix issus des bases FAOSTAT sur les prix producteur et sur les quantités totales et en valeurs échangées, pour chaque culture et chaque pays en désagrégé. Les prix donnés par FAOSTAT sont à un niveau de désagrégation inférieur à celui des données disponibles avec FABIO, pour simplifier l'exposé nous nommerons ce niveau \textit{item}. Une table de passage nous permet de passer de ces ?? \textit{item} à ?? \textit{cbs}.

L'utilisation de ces deux bases permet de recouvrir le plus de prix possible. Afin de s'assurer de la cohérence des prix, nous utilisons les données de production totale en quantité et en valeur des \textit{items}, agrégées au niveau monde, ainsi nous récupérons un prix moyen pour chaque culture correspondant à la valeur totale produite sur la quantité totale produite, et nous assurons que les prix issus des données producteurs et des données de commerce, dans chaque pays ne sont pas trop éloignés de cette valeur. La procédure de conservation des prix est la suivante~:
\begin{itemize}
    \item utiliser les {80\,\%} des prix d'\textit{item}, les plus proches du prix moyen mondial trouvé~;
    \item calculer la moyenne et l'écart type de la répartition de ces prix~;
    \item calculer pour l'ensemble des prix (pas uniquement les 80\,\% des prix les plus proches), l'écart à la moyenne en nombre d’écart-type~;
    \item calculer pour chaque pays l'écart-type moyen pour chaque pays, afin de déterminer si le pays a des prix habituellement élevés par rapport au prix mondiaux~;
    \item ne conserver que les prix ne s'éloignant pas de plus de deux écart-types en plus de leur écart moyen, calculé à l'étape précédente, des prix mondiaux.
\end{itemize}
On applique également ce tri, sur les prix agrégés au niveau supérieur des \textit{cbs}. Ensuite, on agrège les prix restant au niveau des \textit{cbs}.

% [expliquer ces affaires de exports imports]


\subsection{Quantités - FABIO}

Les données relatives aux quantités produites, consommées et échangées proviennent du modèle FABIO. Les données brutes\footnote{cf. annexe de \cite{Bruckner2019}, pour voir des représentations graphiques des matrices.} étaient sous la forme de matrice de production $x$ (de dimensions $1 \times (card(pays)+card(biens))$), de demande finale $y$ (de dimension $(card(pays)+card(biens)) \times (card(pays)+card(uses))$) et d'échanges $z$ (et de dimensions $(card(pays)+card(biens)) \times (card(pays)+card(biens))$). Nous avons retiré certains secteurs qui ne nous intéressaient pas (alcools et éthanol) représentant ??~\% de la production totale, la production de \textit{Sweeteners, Other} représentant ??~\%, ainsi que la production d'huile et de tourteaux provenant de céréales représentant ??~\%. Pour éviter d'avoir une région \textit{reste du monde} inconsistante nous l'avons retirée (??~\%).

Nous avons ensuite converti ces valeurs en quantité avec les prix issus du traitement de la section \ref{subsec:prix}, afin d'éviter tout problème d'équilibre, dû à une inconsistance des prix, nous avons réévalué, pays par pays, les prix des outputs de certains processus à la hausse pour permettre un équilibre entre les coûts des inputs et la valeur des outputs produits. Pour éviter toute perturbation entre des processus consécutifs sur une même chaîne de valeur, nous avons fait ce traitement dans l'ordre de ces chaînes.

En parallèle, nous avons créé une table de correspondance des processus à partir des processus existants dans la matrice $z$, privée des auto-consommation correspondant à des pertes ou à un usage pour la production de semences (ayant fait le choix de ne pas prendre en compte ces possibilités dans les équations du modèle)\footnote{D'après \cite{Bruckner2019}, les quantités utilisées pour la production de semences représentent 1,4~\% de la consommation des biens issus de culture}. Cette matrice référence tous les processus existants réellement dans les données, nous nous sommes basé$\cdot$e$\cdot$s sur les processus proposés dans FABIO, en considérants néanmoins tous les laits comme étant issus du même processus. Cette table nous permet de créer ensuite deux matrices d'usage et de production, telles que $use$ (dimensions ) représente l'ensemble des flux de biens vers des processus, et $make$ (dimensions ) représente l'ensemble des productions issues de chaque processus. Ces matrices donnent ainsi plus d'informations une fois les pays et les biens agrégés, que les matrices agrégées $x$ et $z$\footnote{Nous n'avons pas utilisé les matrices $make$ and $use$ de FABIO, car celles-ci n'étaient pas équilibrées avec les autres matrices $x$, $y$ et $z$}.

Nous avons ensuite agrégés les matrices selon notre choix de régions et de biens agricoles, tout en vaillant à conserver des valeurs positives dans chaque processus, et en empêchant illusions de pertes et de semences engendrées par l'agrégation.

Pour certaines régions, même après agrégations, certains processus étaient incomplets, i.e. un processus produisait des outputs, sans consommer d'inputs, ou bien créer qu'un seul output, sur les multiples, nécessaires. Nous avons donc fait le choix de soustraire à nos matrices, la production, et la consommation des éléments intervenant dans ces processus (??~\%).

Finalement, nous avons à partir de nos matrices agrégées $use$, $make$ et $y$, l'ensemble des données nécessaires pour le modèle, vis-à-vis des consommations $C$, des productions $Q$, et des usages intermédiaires $x$, $X$. FABIO fournissant également des données relatifs aux surfaces, nous avons aussi les informations relatives à $s$, $L$ et $y$. Ces données sont en valeurs, cependant étant donné que nous étudions les variations, l'ensemble des prix est posé à 1, ce qui nous permet également d'avoir les informations sur les prix $P$.

Le tableau \ref{tab:prod_item} montre la production en valeurs des différents biens retenus. L'annexe \ref{annexe:item_prc} présente la composition des biens, et leur origine et usage, dans les différents procédés.


\begin{table}[h]
    \centering
    \begin{threeparttable}
        % Premier mini-tableau (colonne gauche)
        \begin{minipage}[t]{0.49\textwidth}
            \centering
            \begin{tabularx}{\textwidth}{p{1.8in}c}
                \textbf{Biens} & \textbf{Part (\%)} \\ \hline
                % Première moitié de vos données
                \csname @@input\endcsname /home/croch/Documents/Stage/Model/Git_ag_policies/ag-policies-ghg/tab_latex/outputs/prod_item_part1.tex
                \hline
            \end{tabularx}
        \end{minipage}
        \begin{minipage}[t]{0.49\textwidth}
            \centering
            \begin{tabularx}{\textwidth}{p{2in}c}
                \textbf{Biens} & \textbf{Part (\%)}                                                                                               \\ \hline
                % Seconde moitié de vos données
                \csname @@input\endcsname /home/croch/Documents/Stage/Model/Git_ag_policies/ag-policies-ghg/tab_latex/outputs/prod_item_part2.tex \\
                \hline
            \end{tabularx}
        \end{minipage}
        \caption{Allocation des émissions}
        \label{tab:prod_item}
    \end{threeparttable}
\end{table}

\subsection{Droits de douanes - MAcMap}

Nous avons récupéré les données relatives aux droits de douanes mises à dispositions par \cite{Guimbard2012}, via la base de données MAcMap-HS6 (pour Market Access Map, selon la classification SH6) les données ne concernent cependant pas l'année 2017, nous avons donc pris les données de l'année 2016. Cette base rend compte des droits de douanes bilatéraux d'environ 190 pays importateurs et 220 exportateurs, sur 5\,000 produits. Le traitement a donc consisté en une agrégation selon nos régions et nos biens, en pondérant les droits de douanes par les quantités échangées. Pour les valeurs manquantes dans certains pays, nous les avons émulées en considérant que ces pays échangés dans les mêmes proportions que le monde entier. Ce traitement nous permet donc d'accéder aux valeurs de $T$ dans notre modèle. La table \ref{tab:macmap}, expose certains des droits de douane existants.


\subsection{Usage d'énergie et usage d'entrants - GTAP}

Nous avons récupéré de la base de donnée GTAP 11\footcite{Aguiar2022}, les informations concernant le coût de la terre, l'usage de fertilisants (nous avons réalloué ceux alloués aux bêtes, au fourrage, étant donné que dans notre modèle, les animaux, n'occupent de l'espace que par ce qu'ils mangent), ainsi que l'utilisation d'énergie à la ferme (que nous utilisons ensuite dans le traitement des émissions). Ce traitement nous permet donc d'avoir les informations relatives à prix de la terre et à l'usage d'entrants, soient $r$ et $F$ dans notre modèle.


\subsection{Émissions de GES - FAOSTAT}

Les données GTAP, nous permettent de traiter les informations relatives aux émissions. Nous avons suivi l'allocation des émissions proposée dans \cite{Valin2023}. Le tableau \ref{tab:em_allocation}, représente cette allocation, et le pourcentage sur les émissions totales.

\begin{table}[h]
    \centering
    \begin{threeparttable}
        \begin{tabularx}{\textwidth}{p{1.5in}p{1.2in}p{1.2in}c}
            \textbf{Catégorie} & \textbf{Gaz} & \textbf{Allocation} & \textbf{Part des émissions}\tnote{a} (en \%) \\ \hline
            \csname @@input\endcsname /home/croch/Documents/Stage/Model/Git_ag_policies/ag-policies-ghg/tab_latex/outputs/em_alloc.tex
            \hline
        \end{tabularx}
        \begin{tablenotes}
            \footnotesize
            \item[a] En utilisant les potentiels de réchauffements présentés dans le rapport AR6 du GIEC, i.e. 27 pour CH\textsubscript{4} et 273 pour N\textsubscript{2}O. Les pourcentages ne somment pas à 100 à cause des arrondis.
        \end{tablenotes}
        \caption{Allocation des émissions}
        \label{tab:em_allocation}
    \end{threeparttable}
\end{table}

Parallèlement, le tableau \ref{tab:em_reg} suivant représente les émissions par régions. \textbf{reprendre pour mettre prod et conso, avec \%}

\begin{table}[h]
    \centering
    \begin{threeparttable}
        \begin{tabularx}{\textwidth}{p{1.9in}cc}
            \textbf{Régions} & \textbf{Émissions totales} (en Mt eqCO\textsubscript{2}) & \textbf{Part des émissions}\tnote{a} (en \%) \\ \hline
            \csname @@input\endcsname /home/croch/Documents/Stage/Model/Git_ag_policies/ag-policies-ghg/tab_latex/outputs/em_reg.tex
            \hline
        \end{tabularx}
        \begin{tablenotes}
            \footnotesize
            \item[a] La somme des parts ne fait pas 100~\%, car nous avons retiré \textit{a posteriori} les émissions du reste du monde, cela signifi que 3~\% des émissions ne sont pas représentées dû à notre choix de retirer cette dernière région.
        \end{tablenotes}
        \caption{Allocation des émissions}
        \label{tab:em_reg}
    \end{threeparttable}
\end{table}

\subsection{Dépense nationale brute - WDI}

Ici nous avons simplement récupéré les données de la banque mondiale relative aux dépenses de consommation finale en dollars constants\footnote{référencées NE.CON.TOTL.CD par la banque mondiale}. Ce qui nous permet de donner une valeur à $E$ dans notre modèle.


\section{Paramètres de comportement}
Nous faisons les mêmes choix sur le paramétrage que dans \cite{Gouel2025}. Le tableau \ref{tab:ela} résume les valeurs choisies pour les élasticités, ainsi que l'origine de ces choix.

\begin{table}[h]
    \centering
    \begin{tabularx}{\textwidth}{l >{\raggedright\arraybackslash}X >{\raggedright\arraybackslash}X}
        \textbf{Élasticité}        & \textbf{Description}                                           & \textbf{Origine}                                   \\ \hline
        $\varepsilon$ = 0,5        & opposé de prix de la demande pour le panier de biens agricoles & \cite{Comin2021}                                   \\
        $\kappa =$ 0,6             & de substitution à la consommation                              & Valeur usuelle dans la littérature \cite{Rude2000} \\
        $\kappa_\text{feed} =$ 0,9 & de substitution dans l'alimentation animale                    & \cite{Rude2000}                                    \\
        $\varsigma_i^k =$ 0,25     & prix rendement des cultures                                    & \cite{Keeney2009}                                  \\
        $\sigma^k \in$ [2,6\,, 10] & de substitution d'Armington                                    & GTAP  \cite{Aguiar2022}, \cite{Costinot2016}       \\
        \hline
    \end{tabularx}
    \caption{Origine des paramètres}
    \label{tab:ela}
\end{table}

Nous pouvons remarquer que l'élasticité de substitution entre les biens pour l'alimentation animale est plus élevée que celle pour l'alimentation humaine, cela témoigne deux choses, d'abord que l'alimentation animale est constitués d'un panel d'aliments moins diversifiés (les animaux mangent principalement des céréales et des tourteaux), ensuite que les humains veulent manger tel ou tel aliment, et non pas que pour son apport calorique.

Le choix de l'élasticité $\varsigma$ vient d'une moyenne des valeurs de l'étude GTAP \cite{Miller2009} qui porte sur les bio-carburants, idéalement la valeur aurait dû être différenciée entre les différentes cultures dans les différents pays pour exprimer les différentes rigidités dans chaque pays.

Les valeurs des élasticités de substitutions d'Armington, viennent quant à elles de la base de données GTAP\footcite{Aguiar2022}, nous avons pris les valeurs disponibles, et pour les produits animaliers non-disponibles, nous avons choisi la valeur de 5,65 pour les produits animaliers correspondant à la valeur affectée pour les animaux vivants, et celle de 5,4 pour les produits transformés d'origine végétale, en se basant sur la valeur calculée sur les dix cultures les plus présentes, dans \cite{Costinot2016}.
