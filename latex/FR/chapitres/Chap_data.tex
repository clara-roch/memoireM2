\section{Données et traitements}
Le modèle est calibré sur des données de 2017, dernière année disponible dans les données GTAP. Comme souligné dans \cite{Gouel2021}, il est important qu'un modèle recouvre le plus de biens agricoles, afin que l'élasticité $\varepsilon$ corresponde réellement à l'inverse de l'élasticité de la demande pour lesdits biens, qui est estimée dans des articles tels que \cite{Comin2021} et non pas seulement à une partie d'entre eux. Nous avons donc choisi de représenter l'ensemble (ou presque) des biens disponibles dans FABIO, en les agrégeant en trente-et-un groupes, qui représentent des éléments uniques ou bien des ensembles. Les biens uniques ont été choisis de telle sorte qu'ils représentent chacun plus d'un pour cent de la production mondiale totale\footnote{cf. table 2 de \cite{Gouel2021}}, les autres étant regroupés par catégories. Quant aux pays, nous avons choisi de les rassembler en quatorze régions, correspondant au regroupement des pays par sous-continents, en évitant d'avoir une région « reste du monde », qui, bien qu'elle apparaisse dans les données FABIO\footnote{Modèle basé sur les données de FAOSTAT, représentant les échanges entre pays, les productions et les usages finaux, regroupant 186 pays et 127 biens agricoles, recouvrant ?? de la production mondiale \cite{Bruckner2019}}, et dans GTAP\footnote{Base de données GTAP, représentant l'économie mondiale, en se basant sur des données de comptabilité nationale, de commerce, de production, etc., répartie en vingt-et-une catégories de biens, et en 160 régions.}, n'a pas de sens géographique ou politique\footnote{L'annexe \ref{annexe:reg} donne l'attribution des pays aux régions.}.


\subsection{Quantités - FABIO}

Les données relatives aux quantités produites, consommées et échangées proviennent du modèle FABIO. Les données brutes\footnote{cf. annexe de \cite{Bruckner2019}, pour voir des représentations graphiques des matrices.} étaient sous la forme d'une matrice de production $x$, d'une demande finale $y$  et d'une matrice d'échanges $z$. Nous avons retiré certains secteurs qui ne nous intéressaient pas (alcools et éthanol) représentant 3,12~\% de la production totale, la production de \textit{Sweeteners, Other} représentant de 0,01~\%, ainsi que la production d'huile et de tourteaux provenant de céréales représentant moins 0,01~\%. Pour éviter d'avoir une région \textit{reste du monde} inconsistante, nous l'avons retirée (pour 0,01~\%).

Nous avons ensuite converti ces valeurs en quantités avec les prix issus du traitement de la section \ref{subsec:prix}. Afin d'éviter tout problème d'équilibre, dû à une inconsistance des prix, nous avons réévalué, pays par pays, les prix des outputs de certains processus à la hausse pour permettre un équilibre entre les coûts des inputs et la valeur des outputs produits. Pour éviter toute perturbation entre des processus consécutifs sur une même chaîne de valeur, nous avons effectué ce traitement dans l'ordre de ces chaînes.

En parallèle, nous avons créé une table de correspondance des processus à partir des processus existants dans la matrice $z$, privée de l'autoconsommation correspondant à des pertes ou à un usage pour la production de semences (ayant fait le choix de ne pas prendre en compte ces possibilités dans les équations du modèle)\footnote{D'après \cite{Bruckner2019}, les quantités utilisées pour la production de semences représentent 1,4~\% de la consommation des biens issus de culture}. Cette matrice référence tous les processus existants réellement dans les données. Nous nous sommes basé$\cdot$e$\cdot$s sur les processus proposés dans FABIO, en considérant néanmoins tous les laits comme étant issus du même processus. Cette table nous permet de créer ensuite deux matrices d'usage et de production, telles que $use$ (dimensions) représente l'ensemble des flux de biens vers des processus, et $make$ (dimensions) représente l'ensemble des productions issues de chaque processus. Ces matrices fournissent ainsi plus d'informations, une fois les pays et les biens agrégés, que les matrices agrégées $x$ et $z$\footnote{Nous n'avons pas utilisé les matrices $make$ et $use$ de FABIO, car celles-ci n'étaient pas équilibrées avec les autres matrices $x$, $y$ et $z$}.

Nous avons ensuite agrégé les matrices selon notre choix de régions et de biens agricoles, tout en veillant à conserver des valeurs positives dans chaque processus et en empêchant les illusions de pertes et de semences engendrées par l'agrégation.

Pour certaines régions, même après agrégation, certains processus étaient incomplets, c'est-à-dire qu'un processus produisait des outputs sans consommer d'inputs, ou bien créait un seul output parmi les multiples nécessaires. Nous avons donc fait le choix de soustraire de nos matrices la production et la consommation des éléments intervenant dans ces processus (représentant 0,01~\% en valeur de la production totale).

Finalement, nous avons, à partir de nos matrices agrégées $use$, $make$ et $y$, obtenu l'ensemble des données nécessaires pour le modèle, vis-à-vis des consommations $C$, des productions $Q$, et des usages intermédiaires $x$, $X$. FABIO fournissant également des données relatives aux surfaces, nous disposons aussi des informations relatives à $s$, $L$ et $y$. Ces données sont en valeurs ; cependant, étant donné que nous étudions les variations, l'ensemble des prix est fixé à 1, ce qui nous permet également d'obtenir les informations sur les prix $P$.

Le tableau \ref{tab:prod_item} montre la production en valeurs des différents biens retenus. L'annexe \ref{annexe:item_pord} présente la composition des biens, ainsi que leur origine et usage dans les différents procédés.


\begin{table}[h]
    \centering
    \begin{threeparttable}
        \begin{minipage}[t]{0.49\textwidth}
            \centering
            \begin{tabularx}{\textwidth}{p{1.8in}c}
                \textbf{Biens} & \textbf{Part} (\%) \\ \hline
                \csname @@input\endcsname /home/croch/Documents/Stage/Model/Git_ag_policies/ag-policies-ghg/outputs_latex/outputs/prod_item_part1.tex
                \hline
            \end{tabularx}
        \end{minipage}
        \begin{minipage}[t]{0.49\textwidth}
            \centering
            \begin{tabularx}{\textwidth}{p{2in}c}
                \textbf{Biens} & \textbf{Part} (\%)                                                                                                   \\ \hline
                \csname @@input\endcsname /home/croch/Documents/Stage/Model/Git_ag_policies/ag-policies-ghg/outputs_latex/outputs/prod_item_part2.tex \\
                \hline
            \end{tabularx}
        \end{minipage}
        \caption{Allocation des émissions}
        \label{tab:prod_item}
    \end{threeparttable}
\end{table}


\subsection{Prix - FAOSTAT \textbf{-- à reprendre}}\label{subsec:prix}

Afin de convertir nos données de volumes en données de valeurs, nous avons récupéré les prix de FAOSTAT. Nous avions besoin des données de prix producteurs ; cependant, les données FAOSTAT sur ces derniers ne correspondaient pas entièrement aux biens présents dans les tables FABIO, il nous fallut donc les compléter avec des données de prix issues des tables de commerce\footnote{Table \textit{Producer Prices} et \textit{Crops and livestock products}}.

Certaines données de prix étaient incohérentes et entraînaient un déséquilibre des chaînes de valeur une fois l'agrégation réalisée avec les données FABIO ; il a donc fallu les traiter afin de ne conserver que des prix pertinents. Tout d'abord, nous avons calculé, pour chaque bien à l'échelle FAOSTAT, une valeur moyenne en divisant la valeur totale échangée par la quantité totale échangée. Nous nous assurons ensuite que les prix issus des données FAOSTAT (tant producteurs que commerciales) dans chaque pays ne s'écartent pas trop de cette valeur. La procédure de conservation des prix est la suivante :
\begin{itemize}
    \item utiliser les 80\,\% des prix d'\textit{item}, les plus proches du prix moyen mondial trouvé, pour calculer une moyenne et un écart-type correspondant à une répartition plus resserrée des prix ;
    \item calculer, pour l'ensemble des prix (et non uniquement les 80\,\% les plus proches), l'écart par rapport à la moyenne en nombre d’écarts-types ;
    \item calculer, pour chaque pays, l'écart-type moyen, afin de déterminer si le pays présente des prix habituellement élevés par rapport au prix mondial ;
    \item ne conserver que les prix s'écartant de moins de deux fois l'écart-type moyen spécifique à chaque pays par rapport au prix mondial.
\end{itemize}
Nous effectuons ce tri sur les données brutes de FAOSTAT, puis sur celles obtenues après agrégation des données conservées lors de ce premier tri, au niveau FABIO. Cette fois-ci, nous nous assurons qu'après agrégation, les prix sont toujours suffisamment proches de la moyenne calculée pour ce niveau, en prenant en compte l'ensemble des valeurs et des quantités échangées.

Finalement, ces deux étapes de traitement des prix conduisent à un taux de prix manquants de 63\,\%. Nous reconstruisons donc les prix à l'aide d'un multiplicateur calculé par pays et par groupe de produits (tel que défini dans la colonne de gauche de l'annexe \ref{annexe:item_pord}). Ce multiplicateur sera plus élevé pour un pays ayant, en moyenne, des prix plus élevés, et également pour les biens dont le prix est supérieur à la moyenne. Ainsi, le multiplicateur le plus élevé correspond au bœuf au Japon. Nous reconstruisons ensuite les prix manquants à l'échelle FABIO en récupérant le premier prix non nul obtenu en multipliant notre multiplicateur par le prix moyen présent dans nos données du bien à l'échelle FABIO, puis par le prix moyen à l'échelle du bien de notre modèle et enfin par le prix moyen mondial de ce bien à l'échelle du modèle. \textbf{Revenir sur ce paragraphe qui est sûrement incompréhensible\dots}


\subsection{Droits de douanes - MAcMap}

Nous avons récupéré les données relatives aux droits de douanes mises à dispositions par \cite{Guimbard2012}, via la base de données MAcMap-HS6 (pour Market Access Map, selon la classification SH6) les données ne concernent cependant pas l'année 2017, nous avons donc pris les données de l'année 2016. Cette base rend compte des droits de douanes bilatéraux d'environ 190 pays importateurs et 220 exportateurs, sur 5\,000 produits. Le traitement a donc consisté en une agrégation selon nos régions et nos biens, en pondérant les droits de douanes par les quantités échangées. Pour les valeurs manquantes dans certains pays, nous les avons émulées en considérant que ces pays échangés dans les mêmes proportions que le monde entier. Ce traitement nous permet donc d'accéder aux valeurs de $T$ dans notre modèle. Les figures \ref{fig:macmap} montre les valeurs de droits de douanes moyens entre les régions et par produits.

\begin{figure}
    \centering
    \begin{subfigure}[b]{0.495\textwidth}
        \centering
        \includegraphics[width=\textwidth]{/home/croch/Documents/Stage/Model/Git_ag_policies/ag-policies-ghg/outputs_latex/outputs/reg_reg_tariff.pdf}
        \caption{Entre pays}
        \label{fig:reg_reg}
    \end{subfigure}
    \begin{subfigure}[b]{0.495\textwidth}
        \centering
        \includegraphics[width=\textwidth]{/home/croch/Documents/Stage/Model/Git_ag_policies/ag-policies-ghg/outputs_latex/outputs/item_reg_tariff.pdf}
        \caption{Différents biens}
        \label{fig:item_reg}
    \end{subfigure}
    \caption{Carte de chaleur des droits de douanes}
    \label{fig:macmap}
\end{figure}

Comme on peut le voir sur la sous-figure de gauche \ref{fig:reg_reg}, il y a des droits de douanes et des quantités importées au sein même de nos régions, cela est dû au fait que nos régions sont des agrégats, ainsi ici seule la Chine (CHN) qui est notre seule région-pays, n'a pas de droits de douane en elle. Les abréviations des régions et des biens sont disponibles en annexes.


\subsection{Usage d'énergie et usage d'entrants - GTAP}

Nous avons récupéré de la base de donnée GTAP 11\footcite{Aguiar2022}, les informations concernant le coût de la terre, l'usage de fertilisants (nous avons réalloué ceux alloués aux bêtes, au fourrage, étant donné que dans notre modèle, les animaux, n'occupent de l'espace que par ce qu'ils mangent), ainsi que l'utilisation d'énergie à la ferme (que nous utilisons ensuite dans le traitement des émissions). Ce traitement nous permet donc d'avoir les informations relatives à prix de la terre et à l'usage d'entrants, soient $r$ et $F$ dans notre modèle.


\subsection{Émissions de GES - FAOSTAT}

Les données GTAP, nous permettent de traiter les informations relatives aux émissions. Nous avons suivi l'allocation des émissions proposée dans \cite{Valin2023}. Le tableau \ref{tab:em_allocation}, représente cette allocation, et le pourcentage sur les émissions totales.
\begin{table}[h]
    \centering
    \begin{threeparttable}
        \begin{tabularx}{\textwidth}{p{1.5in}p{1.2in}p{1.2in}c}
            \textbf{Catégorie} & \textbf{Gaz} & \textbf{Allocation} & \textbf{Part des émissions}\tnote{a} (en \%) \\ \hline
            \csname @@input\endcsname /home/croch/Documents/Stage/Model/Git_ag_policies/ag-policies-ghg/outputs_latex/outputs/em_alloc.tex
            \hline
        \end{tabularx}
        \begin{tablenotes}
            \footnotesize
            \item[a] En utilisant les potentiels de réchauffements présentés dans le rapport AR6 du GIEC, i.e. 27 pour CH\textsubscript{4} et 273 pour N\textsubscript{2}O. Les pourcentages ne somment pas à 100 à cause des arrondis.
        \end{tablenotes}
        \caption{Allocation des émissions}
        \label{tab:em_allocation}
    \end{threeparttable}
\end{table}

Parallèlement, le tableau \ref{tab:em_reg} suivant représente les émissions par régions. \textbf{Reprendre le tableau pour mettre prod et conso (le pb, c'est que données en valeurs\dots), avec \%}

\begin{table}[h]
    \centering
    \begin{threeparttable}
        \begin{tabularx}{\textwidth}{p{1.9in}cc}
            \textbf{Régions} & \textbf{Émissions totales} (en Mt eqCO\textsubscript{2}) & \textbf{Part des émissions}\tnote{a} (en \%) \\ \hline
            \csname @@input\endcsname /home/croch/Documents/Stage/Model/Git_ag_policies/ag-policies-ghg/outputs_latex/outputs/em_reg.tex
            \hline
        \end{tabularx}
        \begin{tablenotes}
            \footnotesize
            \item[a] La somme des parts ne fait pas 100~\%, car nous avons retiré \textit{a posteriori} les émissions du reste du monde, cela signifi que 3~\% des émissions ne sont pas représentées dû à notre choix de retirer cette dernière région.
        \end{tablenotes}
        \caption{Allocation des émissions}
        \label{tab:em_reg}
    \end{threeparttable}
\end{table}

\subsection{Dépense nationale brute - Banque mondiale}

Ici nous avons simplement récupéré les données de la banque mondiale relative aux dépenses de consommation finale en dollars constants\footnote{référencées NE.CON.TOTL.CD par la banque mondiale}. Ce qui nous permet de donner une valeur à $E$ dans notre modèle.


\section{Paramètres de comportement}
Nous faisons les mêmes choix sur le paramétrage que dans \cite{Gouel2025}. Le tableau \ref{tab:ela} résume les valeurs choisies pour les élasticités, ainsi que l'origine de ces choix.

\begin{table}[h]
    \centering
    \begin{tabularx}{\textwidth}{l >{\raggedright\arraybackslash}X >{\raggedright\arraybackslash}X}
        \textbf{Élasticité}        & \textbf{Description}                                           & \textbf{Origine}                                   \\ \hline
        $\varepsilon$ = 0,5        & opposé de prix de la demande pour le panier de biens agricoles & \cite{Comin2021}                                   \\
        $\kappa =$ 0,6             & de substitution à la consommation                              & Valeur usuelle dans la littérature \cite{Rude2000} \\
        $\kappa_\text{feed} =$ 0,9 & de substitution dans l'alimentation animale                    & \cite{Rude2000}                                    \\
        $\varsigma_i^k =$ 0,25     & prix rendement des cultures                                    & \cite{Keeney2009}                                  \\
        $\sigma^k \in$ [2,6\,, 10] & de substitution d'Armington                                    & GTAP  \cite{Aguiar2022}, \cite{Costinot2016}       \\
        \hline
    \end{tabularx}
    \caption{Origine des paramètres}
    \label{tab:ela}
\end{table}

Nous pouvons remarquer que l'élasticité de substitution entre les biens pour l'alimentation animale est plus élevée que celle pour l'alimentation humaine, cela témoigne deux choses, d'abord que l'alimentation animale est constitués d'un panel d'aliments moins diversifiés (les animaux mangent principalement des céréales et des tourteaux), ensuite que les humains veulent manger tel ou tel aliment, et non pas que pour son apport calorique.
Le choix de l'élasticité $\varsigma$ repose sur une moyenne des valeurs issues de l'étude GTAP \cite{Miller2009} portant sur les biocarburants. Idéalement, cette valeur devrait être différenciée selon les cultures et les pays afin de refléter les variations de rigidités.

Les valeurs des élasticités de substitution d'Armington proviennent de la base de données GTAP\footcite{Aguiar2022}. Pour les produits animaliers non disponibles, nous avons retenu une valeur de 5,65 pour les animaux vivants, et celle de 5,4 pour les produits transformés d'origine végétale, en nous appuyant sur la moyenne calculée pour les dix cultures les plus représentées dans \cite{Costinot2016}.
