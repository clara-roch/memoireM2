
Agriculture, responsible for about a quarter of global greenhouse gas (GHG) emissions, is influenced by trade policies, which alter exchange patterns, specialization, and the location of agricultural activities. Tariffs, by protecting certain sectors or limiting imports, have indirect and sometimes counterintuitive effects on emissions by disrupting market equilibria and shaping production and consumption choices. This thesis presents a partial equilibrium model of agricultural trade, accounting for intermediate consumption and the intensive use of inputs, but not changes in total agricultural land area. The data used come from the FABIO project for production and trade flows, FAOSTAT for prices, MAcMap for tariffs, and GTAP for emissions and input use. The model is calibrated for the year 2017 and covers 14 regions and 30 agricultural products, enabling a detailed representation of interactions among actors and products.
Simulating a world without agricultural tariffs shows a 6.1~\% increase in total sectoral GHG emissions. This rise is mainly due to lower costs for animal feed and livestock products, which stimulates their consumption and production—both of which are among the highest emitters. A disaggregated analysis reveals contrasting effects across regions and products: net meat-importing countries see a decrease in local emissions, while exporting countries experience an increase, reflecting greater specialization in livestock production. The removal of tariffs on plant-based products like rice yields more nuanced results, with marginal emission reductions in some cases, particularly when these products replace higher-emission foods in human or animal diets.
In conclusion, tariffs emerge as an ambivalent lever for reducing agricultural emissions, with effects that depend heavily on regional and sectoral contexts. Their elimination, while often touted as economically beneficial, could—without complementary measures—exacerbate the climate impact of the agricultural sector.

\textbf{Keywords:} international trade, agriculture, climate change, tariffs