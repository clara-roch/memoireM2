Cette section présente des premières intuitions sur comment les émissions de gaz à effet de serre (GES) réagissent à l'implémentation de deux politiques agricole~: les droits de douanes et les subventions à la production.

Pour se faire considérons un marché à deux pays, avec un pays importateur $H$ et un pays exportateur $F$.

Nous désignons les fonctions d'offre et de demande pour les deux pays, avec le pays $i \in \{H, F\}$, comme suit~:
$$
    S_i = S_i^0\left(1 + \eta_i\frac{P_i - P_i^0}{P_i^0}\right), \qquad
    D_i = D_i^0\left(1 + \varepsilon_i\frac{P_i - P_i^0}{P_i^0}\right),
$$
où $S_i$ et $D_i$ représentent respectivement les quantités produites et demandées par le pays $i$, $P_i$ est le prix dans le pays $i$, et $\eta_i$ ainsi que $\varepsilon_i$ sont les élasticités de l'offre et de la demande dans le pays $i$. Ici, $X^0$ désigne la valeur initiale de $X$.

Étant donné que les pays constituent l'entièreté de l'économie, la différence entre la demande et la production dans un pays est égale à la différence à l'inverse de celle du pays extérieur, ainsi~:
$$
    D_H - S_H = S_F - D_F.
$$

Pour simplifier la suite des calculs, nous introduisons les élasticités agrégées suivantes~:
\begin{itemize}
    \item élasticité de demande totale
          $$
              \varepsilon = \frac{\partial D}{\partial P_F} \frac{P_F^0}{D^0} = \left( \varepsilon_H \frac{D_H^0}{P_H^0} + \varepsilon_F \frac{D_F^0}{P_F^0} \right)\frac{P_F^0}{D^0} < 0,
          $$
    \item élasticité d'offre totale
          $$
              \eta = \frac{\partial S}{\partial P_F} \frac{P_F^0}{S^0} = \left( \eta_H \frac{S_H^0}{P_H^0} + \eta_F \frac{S_F^0}{P_F^0} \right)\frac{P_F^0}{S^0} > 0,
          $$
    \item élasticité de la demande d'importation domestique
          $$
              \mu_H = \frac{\partial (D_H - S_H)}{\partial P_H} \frac{P_H^0}{M_H^0} = \frac{\varepsilon_H D_H^0 - \eta_H S_H^0}{M_H^0} < 0,
          $$
    \item élasticité de l'offre à l'exportation étrangère
          $$
              \chi_F = \frac{\partial (S_F - D_F)}{\partial P_F} \frac{P_F^0}{X_F^0} = \frac{\eta_F S_F^0 - \varepsilon_F D_F^0}{X_F^0} > 0.
          $$
\end{itemize}

Pour chaque politique on examine ses effets sur les émissions totales à travers leur impact sur les prix internationaux (prix du pays $F$) et sur la production totale.

Considérons que le pays $H$ met en place un droit de douane à l'importation $t$. Cela implique les relations suivantes~:
$$
    P_H = P_F + t.
$$

Sous la politique douanière, les prix dans le pays exportateur deviennent
$$
    \frac{P_F}{P_F^0} = -\frac{\mu_H (1 - t/P_H^0) - \chi_F}{\eta - \varepsilon}\frac{X_F^0}{D^0},
$$
et varient négativement selon $t$~:
$$
    \frac{\partial P_F}{\partial t} = \frac{\mu_H}{\eta - \varepsilon} \frac{X_F^0}{D^0} \frac{P_F^0}{P_H^0} < 0.
$$

La production totale des deux pays est donnée par
$$
    Q = S_H^0 + S_F^0 + \frac{(P_H^0 - P_F^0 - t)(S_H^0 \eta_H \chi_F + S_F^0 \eta_F \mu_H)}{P_F^0 \mu_H - P_H^0 \chi_F},
$$
et varie selon
$$
    \frac{\partial Q}{\partial t} = \frac{S_H^0 \eta_H \chi_F + S_F^0 \eta_F \mu_H}{P_F^0 \mu_H - P_H^0 \chi_F}.
$$
Le signe de ce changement est déterminé par $S_H^0 \eta_H \chi_F + S_F^0 \eta_F \mu_H$. Il n'y a donc pas d'effet clair des tarifs douaniers sur la production totale~: un premier effet (direct) augmente la production dans le pays $H$, tandis qu'un second (indirect) réduit la production totale par la baisse des prix extérieurs.

Concernant les émissions totales $E$, si on considère que les émissions évoluent linéairement avec la production, on obtient~:
$$
    E = E^0 + \frac{(P_H^0 - P_F^0 - t)(E_H^0 \eta_H \chi_F + E_F^0 \eta_F \mu_H)}{P_F^0 \mu_H - P_H^0 \chi_F},
$$
et donc
$$
    \frac{\partial E}{\partial t} = \frac{E_H^0 \eta_H \chi_F + E_F^0 \eta_F \mu_H}{P_F^0 \mu_H - P_H^0 \chi_F}.
$$
Ici, le signe est le même que celui de $E_H^0 \eta_H \chi_F + E_F^0 \eta_F \mu_H$. Autrement dit, l'effet de l'augmentation des tarifs douaniers sur les émissions totales est ambigu~; des émissions nationales plus importantes $E_H^0$ augmentent la probabilité que l'augmentation des droits de douanes augmente les émissions globales.

Voir annexe \ref{appendix:intuitions_tariff} pour le détail des calculs et pour des cas particuliers.
